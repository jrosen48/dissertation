\documentclass[]{book}
\usepackage{lmodern}
\usepackage{amssymb,amsmath}
\usepackage{ifxetex,ifluatex}
\usepackage{fixltx2e} % provides \textsubscript
\ifnum 0\ifxetex 1\fi\ifluatex 1\fi=0 % if pdftex
  \usepackage[T1]{fontenc}
  \usepackage[utf8]{inputenc}
\else % if luatex or xelatex
  \ifxetex
    \usepackage{mathspec}
  \else
    \usepackage{fontspec}
  \fi
  \defaultfontfeatures{Ligatures=TeX,Scale=MatchLowercase}
\fi
% use upquote if available, for straight quotes in verbatim environments
\IfFileExists{upquote.sty}{\usepackage{upquote}}{}
% use microtype if available
\IfFileExists{microtype.sty}{%
\usepackage{microtype}
\UseMicrotypeSet[protrusion]{basicmath} % disable protrusion for tt fonts
}{}
\usepackage[margin=1in]{geometry}
\usepackage{hyperref}
\hypersetup{unicode=true,
            pdftitle={Examining youth engagement during learning activities that involve work with data: An Experience Sampling approach},
            pdfauthor={Joshua M. Rosenberg},
            pdfborder={0 0 0},
            breaklinks=true}
\urlstyle{same}  % don't use monospace font for urls
\usepackage{natbib}
\bibliographystyle{apalike}
\usepackage{longtable,booktabs}
\usepackage{graphicx,grffile}
\makeatletter
\def\maxwidth{\ifdim\Gin@nat@width>\linewidth\linewidth\else\Gin@nat@width\fi}
\def\maxheight{\ifdim\Gin@nat@height>\textheight\textheight\else\Gin@nat@height\fi}
\makeatother
% Scale images if necessary, so that they will not overflow the page
% margins by default, and it is still possible to overwrite the defaults
% using explicit options in \includegraphics[width, height, ...]{}
\setkeys{Gin}{width=\maxwidth,height=\maxheight,keepaspectratio}
\IfFileExists{parskip.sty}{%
\usepackage{parskip}
}{% else
\setlength{\parindent}{0pt}
\setlength{\parskip}{6pt plus 2pt minus 1pt}
}
\setlength{\emergencystretch}{3em}  % prevent overfull lines
\providecommand{\tightlist}{%
  \setlength{\itemsep}{0pt}\setlength{\parskip}{0pt}}
\setcounter{secnumdepth}{5}
% Redefines (sub)paragraphs to behave more like sections
\ifx\paragraph\undefined\else
\let\oldparagraph\paragraph
\renewcommand{\paragraph}[1]{\oldparagraph{#1}\mbox{}}
\fi
\ifx\subparagraph\undefined\else
\let\oldsubparagraph\subparagraph
\renewcommand{\subparagraph}[1]{\oldsubparagraph{#1}\mbox{}}
\fi

%%% Use protect on footnotes to avoid problems with footnotes in titles
\let\rmarkdownfootnote\footnote%
\def\footnote{\protect\rmarkdownfootnote}

%%% Change title format to be more compact
\usepackage{titling}

% Create subtitle command for use in maketitle
\newcommand{\subtitle}[1]{
  \posttitle{
    \begin{center}\large#1\end{center}
    }
}

\setlength{\droptitle}{-2em}
  \title{Examining youth engagement during learning activities that involve work
with data: An Experience Sampling approach}
  \pretitle{\vspace{\droptitle}\centering\huge}
  \posttitle{\par}
  \author{Joshua M. Rosenberg}
  \preauthor{\centering\large\emph}
  \postauthor{\par}
  \predate{\centering\large\emph}
  \postdate{\par}
  \date{2018-05-09}

\usepackage{booktabs}
\usepackage{longtable}
\usepackage{booktabs}
\usepackage{longtable}
\usepackage{array}
\usepackage{multirow}
\usepackage[table]{xcolor}
\usepackage{wrapfig}
\usepackage{float}
\usepackage{colortbl}
\usepackage{pdflscape}
\usepackage{tabu}
\usepackage{threeparttable}
\usepackage[normalem]{ulem}

\usepackage{amsthm}
\newtheorem{theorem}{Theorem}[chapter]
\newtheorem{lemma}{Lemma}[chapter]
\theoremstyle{definition}
\newtheorem{definition}{Definition}[chapter]
\newtheorem{corollary}{Corollary}[chapter]
\newtheorem{proposition}{Proposition}[chapter]
\theoremstyle{definition}
\newtheorem{example}{Example}[chapter]
\theoremstyle{definition}
\newtheorem{exercise}{Exercise}[chapter]
\theoremstyle{remark}
\newtheorem*{remark}{Remark}
\newtheorem*{solution}{Solution}
\begin{document}
\maketitle

{
\setcounter{tocdepth}{1}
\tableofcontents
}
\chapter{Introduction}\label{intro-placemarker}

Changes in how citizens plan our day-to-day lives, communicate, and
learn are increasingly impacted by data. These sources of data--either
quantitative \emph{or} qualitative--are created by us, for us, and about
us, although at present opportunities for learners to analyze data in
educational settings remain limited. Work with data includes broad
processes of collecting, creating, modeling data, and eveb asking
questions that can be answered with data.

Working with data, then, is more than just crunching numbers, or
interpreting a figure created by someone else. Rather, it is about
making sense of phenomena in the world (or solving problems), a point
particularly relevant to those interested in the educational plaec and
implications of working with data (Lee \& Wilkerson, in press; Singer,
Hilton, \& Schweingruber, 2006; Wild \& Pfannkuch, 1999). Aspects of
work with data cut across STEM domains and are recognized as core
competencies across recent curricular documents. For example, the
\emph{Next Generation Science Standards} and the \emph{Common Core State
Standards} (in mathematics; National Governors Association Center for
Best Practices, Council of Chief State School Officers, 2010; NGSS Lead
States, 2013) both highlight the role of authentic work work with data.
Scholars have pointed out the benefits of working with data for learners
as young as two years old (Gopnik, \& Sobel, 2000).

In supporting teachers and learners' data analysis efforts, scholars
have examined a variety of learning-related practices. In particular,
past research has focused on mathematical practices, or activities, like
generating measures of phenomena and creating data models (English,
2012; Lehrer \& Romberg, 1996; Lesh, Middleton, Caylor, \& Gupta, 2008).
Findings from this area of research suggest that engaging in these
practices ``has an exceptionally high payoff in terms of students'
scientific reasoning'' (Lehrer \& Schauble, 2015, p.~696) and can
highlight the utility of mathematics for students' lives (Lesh,
Middleton, Caylor, \& Gupta, 2008).

Because engaging in data-related practices seems to be so potentially
beneficial to learners (Lee \& Wilkerson, in press; National Research
Council, 2012), we need to better understand the nature of learners'
engagement in learning activities that involve various aspects of work
with data. To date, past research shows that using an engagement
framework to characterize students' learning activities is highly
informative. One's engagement in learning tasks is a key outcome in its
own right and may be an antecedent of changes in other outcomes, such as
their well-being, achievement and pursuit of an area of study or career
(Sinatra, Heddy, \& Lombardi, 2015; Wang, Chow, Hofkens, \& Salmela-Aro,
2015; Wang \& Eccles, 2012). However, research has not examined
engagement in these data-related activities in particular.

The purpose of this study, then, is to examine youth engagement in a
variety of learning activities that involve work with data. Engagement
in work with data is explored in the context of outside-of-school STEM
enrichment programs carried out during the summer and work with data is
considered in terms of specific aspects identified from past research,
such as asking questions and generating and modeling data. Knowing more
about how youth engage in work with data is valuable as engagement is a
meaningful outcome for STEM learners in its own right (Sinatra et al.,
2015). Knowing more about how youth engage can also provide a foundation
for subsequent work to explore how particular curricula and engaging
experiences for youth spark their interest in work with data, including
hobbies and occupations related to data science, but also in STEM
domains in general.

\chapter{Literature Review}\label{literature-review}

The framework for this study is informed by work on STEM-related
learning practices, or activities, student engagement, and analytic
approaches to modeling multidimensional constructs. In this review of
literature, I define work with data as a key practice across STEM
domains. I also describe and justify a multi-dimensional framework for
understanding engagement, and then review an approach to analyzing data
that is ideal for capturing this multidimensionality.

\section{Defining Work with Data}\label{defining-work-with-data}

Some scholars have focused on a few key pieces of data analysis,
connected through the use of ``data to solve real problems and to answer
authentic questions'' (Hancock et al., 1992, p.~337). This focus on
solving real problems or answering authentic questions--rather than
being taught and learned as isolated skills--is an essential part of
work with data having the most educational benefits to learners
(National Research Council, 2012; see Lehrer and Schauble {[}2012{]}
Windschitl, Thompson, \& Braaten {[}2018{]} for excellent, practice,
in-depth examples of work with data being used as part of instructional
approaches). This approach has primarily been taken up by mathematics
educators and is reflected in statistics curriculum documents (Franklin
et al., 2007). In science settings, where answering questions about
phenomena serve as the focus of activities, it shares features of the
process of engaging in scientific and engineering practices, but has
been less often studied.

While work with data has been conceived in different ways, some core
components have emerged. See Lee and Wikerson's (in press) forthcoming
summary report for the National Academy of Sciences and Wild and
Pfannkuch (1999), Franklin et al. (2007), and Lehrer and Schauble (2004)
for some specific examples from different domains. The core components,
when synthesized, are better for understanding work with data across
STEM content areas--as in the present study--than the components from
specific examples, which were developed for use in only one domain. The
aspects of work with data that have been articulated in prior studies
are distilled into five key aspects (Figure 1) for use in this study.
They are:

\begin{itemize}
\tightlist
\item
  \emph{Asking questions}: Generating questions that can be answered
  with empirical evidence
\item
  \emph{Making observations}: Watching phenomena and noticing what is
  happening with respect to the phenomena or problem being investigated
\item
  \emph{Generating data}: The process of figuring out how or why to
  inscribe an observation as data about a phenomena, as well as
  generating coding frames or tools for measuring
\item
  \emph{Data modeling}: Activities involving use of simple statistics,
  such as the mean and variance, as well as more complicated models,
  such as linear models and extensions of the linear model
\item
  \emph{Interpreting and communicating findings}: Activities related to
  identifying a driving question regarding the phenomena that the
  question is about
\end{itemize}

\begin{figure}

{\centering \includegraphics[width=0.8\linewidth]{images/figure1} 

}

\caption{Work with data in STEM education settings}\label{fig:unnamed-chunk-1}
\end{figure}

These five synthesized practices are a part of a cycle because not only
does each part follow that before it, but also because the overall
process is iterative: interpreting findings commonly leads to new
questions and subsequent engagement in work with data. Also, scholars
have pointed out some key features of how work with data is carried out
that impact their effectiveness as a pedagogical approach. These key
features include an emphasis on making sense of real-world phenomena and
iterative cycles of engaging in work with data and collaboration and
dialogue, through which ideas and intermediate findings are critiqued
and subject to critique, and revised over time (McNeill \& Berland,
2017; Lee \& Wilkerson, in press).

\section{The role of working with data in STEM learning
environments}\label{the-role-of-working-with-data-in-stem-learning-environments}

Working with data can serve as an organizing set of practices for
engaging in inquiry in STEM learning settings (Lehrer \& Schauble,
2015). Data are both encountered and generated by learners, and so
opportunities for learners to work with data provide many opportunities
to leverage their curiosity because processes of inquiry can be grounded
in phenomena that learners themselves can see and manipulate or
phenomena that learners are interested in. Also important, becoming
proficient in work with data can provide learners with an in-demand
capability in society, owing to the number of occupations, from
education to entrepreneurship, that demand or involve taking action
based on data (Wilkerson \& Fenwick, 2017). Furthermore, becoming
proficient in work with data can be personally empowering because of the
parts of our lives--from paying energy bills to interpreting news
articles--that use data.

Recent educational reform efforts emphasize work with data (i.e., the
scientific and engineering practices in the NGSS and the standards for
mathematical practice in the Common Core State Standards). However, work
with data is uncommon in many classroom settings (even classrooms
emphasizing recent science education reform effrts; McNeill \& Berland,
2017; Miller, Manz, Russ, Stroupe, \& Berland, in press), and so
learning environments suited to engaging in work with data, but not
explicitly designed to support it, may be valuable to study because they
may serve as incubators of these rare and challenging learning
activities.

Outside-of-school programs are a potentially valuable setting to explore
engagement in work with data because of the combined pedagogical and
technical expertise of their staff and the activities learners do during
their participation in them. Staff for these programs includes educators
and scientists, engineers, and others with the technical experience.
Additionally, the programs were designed to involve learners in the
types of real-world practices experienced by experts in STEM
disciplines. Attendance in such programs is associated with many
benefits to learners (Green, Lee, Constance, \& Hynes, 2013; see Lauer,
Akiba, Wilkerson, Apthorp, Snow, \& Martin-Glenn, 2006, for a
comprehensive review). These programs are also selected because little
research has examined how data are part of the experiences of youth in
out-of-school-time programs, despite its place as one of a few core
practices in STEM. While these reasons to study work with data focus on
outside-of-school programs, they are also germane and applicable to more
formal learning environments, such as classrooms, in which teachers want
to design opportunities for their learners to work with data.

\section{What We Know (And Do Not Know) About How Youth Work with
Data}\label{what-we-know-and-do-not-know-about-how-youth-work-with-data}

Scholars have researched cognitive capabilities as outcomes related to
work with data. Much of this laboratory-based research has focused on
how children develop the capability to inductively reason from
observations (Gelman \& Markman, 1987). Other research has focused on
the development of causal, or mechanistic, reasoning, among young
children (Gopnik et al., 2001; Gopnik \& Sobel, 2000), often from a
Piagetian, individual-development focused tradition (i.e., Piaget \&
Inhelder, 1969). A key outcome of engaging in work with data has to do
with how learners account for variability (Lehrer, Kim, \& Schauble,
2007; Petrosino, Lehrer, \& Schauble, 2003; Lesh, Middleton, Caylor, \&
Gupta, 2008; Lee, Angotti, \& Tarr, 2010), arguably the main goal of
engaging in work with data (Konold \& Pollatsek, 2002). From this
research, we know that learners can develop the capacity to reason about
variability (and covariability).

Past research has also shown that there are strategies that can support
work with data. These include the design of technological tools and the
development of curricula. From this research, we know about specific
strategies and learning progressions for learners to develop this
capability, such as the role of measurement in exposing learners in a
direct way to sources of variability (Petrosino et al., 2003), role of
simulation to learn about sampling distributions (Stohl \& Tarr, 2002),
and use of relevant phenomena, such as manufacturing processes, such as
the size of metallic bolts, which can help learners to focus on
``tracking a process by looking at its output'' (Konold \& Pollatsek,
2002, p.~282).

Finally, past research has shown that different aspects of work with
data pose unique opportunities and challenges. Asking empirical
questions requires experience and ample time to ask a question that is
both able to be answered with data and which is sustaining and worth
investigating (Bielik, 2016; Hasson \& Yarden, 2012). Making
observations and generating data, such as of the height of the school's
flagpole, requires negotiation not only of what to measure, but how and
how many times to measure it (Lehrer, Kim, \& Schauble, 2007). Regarding
modeling, not only teaching students about models, such as that of the
mean, but also asking them to create them, are valuable and practical
(Lehrer \& Schauble, 2004; Lehrer, Kim, \& Jones, 2011), but also
time-intensive. Interpreting findings, especially in light of
variability through models, and communicating answers to questions,
means not only identifying error but understanding its sources, and can
be supported through exploring models that deliberately represent the
data poorly, but can be instructive for probing the benefits and
weaknesses of models (Lee \& Hollebrands, 2008; Lehrer, Kim, \&
Schauble, 2007).

Despite this past research, how learners and youth participate in
different aspects of work with data in terms of engagement theory has
not been examined. Consider the practice of modeling data, commonly
described as a----or \emph{the}----key part of many applied data
analyses (Konold, Finzer, \& Kreetong, 2017). When modeling data,
learners may use data they generated and structured in a data set on
their own, or may model already-processed, or use already-plotted, data
(McNeill \& Berland, 2017). How challenging do students perceive the
different enactments of these activities to be and how do learners
perceive their competence regarding them? Importantly, how hard are
learners working? How much do they feel they are learning? Knowing more
about these beliefs, characteristics, and processes could help us to
develop informed recommendations for teachers and designers intending to
bring about opportunities for learners to engage in work with data in a
better-supported way that is sustained over time.

\section{Engagement in General and in STEM
Domains}\label{engagement-in-general-and-in-stem-domains}

The nature of engagement is discussed in terms of general features that
have been identified across content area domains, conditions that
support engagement, and differences between engagement in general and in
STEM settings. This is followed by a discussion of two key features of
engagement: its dynamic, or context-dependent, characteristics and its
multidimensional nature. Finally, methods for capturing these two
features empirically through the Experience Sampling Method, or ESM--and
how this (multidimensional) data can be analyzed--are described.

Engagement is defined in this study as active involvement, or
investment, in activities (Blumenfeld et al., 2004). Explaining how
learners are involved in activities and tasks is especially important if
we want to know about what aspects of work with data are most engaging
(and in what ways), and therefore can serve as exemplary for others
advancing work with data as well as those calling for greater support
for engagement. Apart from being focused on involvement, engagement is
often thought of as a meta-construct, that is, one that is made up of
other constructs (Skinner \& Pitzer, 2012; Skinner, Kindermann, \&
Furrer, 2009). By defining engagement as a meta-construct, scholars
characterize it in terms of cognitive, behavioral, and affective
dimensions that are distinct yet interrelated (Fredricks, 2016).

We know from past research that the cognitive, behavioral, and affective
dimensions of engagement can be distinguished (Wang \& Eccles, 2012;
Wang \& Holcombe, 2012) and that while there are long-standing concerns
about the conceptual breadth of engagement (Fredricks et al., 2016),
careful justification and thoughtful use of multidimensional engagement
constructs and measures is warranted. Engagement is also considered to
be changing in response to individual, situation or moment contextual
factors, Skinner and Pitzer's (2012) model of motivational dynamics,
highlighting the community, school, classroom, and even learning
activity, shows the context-dependent nature of engagement on the basis
of the impacts of these factors on learners' engagement.

Engagement in STEM settings shares characteristics with engagement
across disciplines, yet there are some distinct aspects of it (Greene,
2015). While one type of engagement---behavioral---is associated with
achievement-related outcomes, many STEM practices call for engagement in
service of other outcomes, especially around epistemic and
agency-related dimensions (Sinatra et al., 2015,). For example, many
scholars have defined scientific and engineering practices as cognitive
practices, which involve applying \emph{epistemic considerations} around
sources of evidence and the nature of explanatory processes (see Berland
et al. 2016, Stroupe, 2014; Miller et al., in press).

The emphasis on developing new knowledge and capabilities through
engaging in STEM practices must be reflected in how the cognitive
dimension of engagement is measured. Because of the importance of
constructing knowledge to engagement in STEM practices, then, cognitive
engagement is defined for this study in terms of learning something new
or getting better at something. While sometimes defined in terms of
extra-curricular involvement or following directions, behavioral
engagement is defined in this study as working hard at and concentrating
on learning-related activities (Fredricks et al., 2004; Singh,
Granville, \& Dika, 2002). Finally, affective engagement is defined as
affective responses to activities, such as being excited, angry, or
relaxed (Pekrun \& Linnenbrink-Garcia, 2012).

\section{The conditions in which engagement may occur and the factors
that impact
engagement}\label{the-conditions-in-which-engagement-may-occur-and-the-factors-that-impact-engagement}

Past research suggests learners or youths' characteristics, such as
their interest in the domain of study, impact their cognitive,
behavioral, and affective engagement (Shernoff et al., 2003; Shernoff et
al., 2016; Shumow, Schmidt, \& Zaleski, 2013). These are both
moment-to-moment, context-dependent conditions in which engagement may
occur, as well as factors at the level of individual differences (i.e.,
youths' more stable interest in STEM domains) that may impact
engagement.

Focusing first on context-dependent conditions, Emergent Motivation
Theory (EMT; Csikszentmihalyi, 1990), provides a useful lens. From EMT,
a key moment-to-moment condition for engagement is how difficult
individuals perceive an activity to be, or its \emph{perceived
challenge}. Another key condition is how good at an activity individuals
perceive themselves to be, or their \emph{perceived
competence}.Important in terms of the EMT, being challenged by and good
at an activity are especially engaging experienced when together. Past
research has supported this contention (Csikszentmihalyi, 1990). As one
empirical example, Shernoff et al. (2016) demonstrated that while
challenge and skill with high levels of one but low levels on the other
(i.e., high challenge and low skill) were not broadly associated with
positive forms of engagement, their interaction was. These findings
suggest that learners' perceptions of the challenge of the activity, and
their perceptions of how skillful they are, are important conditions
that co-occur with learners' engagement. Conceptualizing perceptions of
challenge and competence as conditions, rather than factors that
influence engagement, is in recognition of their co-occurrence within
individuals, in that youth experience engagement and their perceptions
of the activity (perceived challenge) and of themselves (perceive
competence) together and at the same time.

A factor that can support engagement concerns teacher support for
specific practices (Strati, Schmidt, \& Maier, 2017). Particularly
concerning work with data, which is demanding not only for learners but
also teachers, sustained support from teachers is an essential component
of learners being able to work with data (Lehrer \& Schauble, 2015;
Wilkerson, Andrews, Shaban, Laina, \& Gravel, 2016). Consequently, this
study considers work with data through the use of a coding frame that
characterizes the extent to which teachers are supporting specific STEM
practices in their instruction, including aspects of work with data.

Other factors that impact youths' engagement are individual differences.
In recognition of differences among learners in their tendency to engage
in different (higher or lower) ways in specific activities based in part
on individual differences (Hidi \& Renninger, 2006), learners' interest
in STEM before the start of the programs is also considered as a factor
that can impact engagement. Finally, gender and the racial and ethnic
group of students is added, as past research has indicated these as
factors that influence engagement in STEM (Bystydzienski, Eisenhart, \&
Bruning; Shernoff \& Schmidt, 2008).

\section{Challenges of Studying Engagement as a Contextually-Dependent
and Multidimensional
Construct}\label{challenges-of-studying-engagement-as-a-contextually-dependent-and-multidimensional-construct}

Because of the way engagement has been thought of as having
context-dependent characteristics and being multi-dimensional, it is
challenging to use (when conceptualized in such a way) in empirical
studies. One methodological approach that has benefits in terms of both
the context-dependent and multidimensional nature of engagement is the
ESM. A number of scholars have explored or extolled benefits to its use
in their recent work (e.g., Strati et al., 2017; Turner \& Meyer, 2000;
Sinatra et al., 2015).

ESM involves asking---usually using a digital tool and occasionally a
diary---to ask participants short questions about their experiences. ESM
is particularly well-suited to understanding the context-dependent
nature of engagement because students answered brief surveys about their
experience when they were signaled, minimally interrupting them from the
activity they are engaged in and also seeking to collect measures about
learners' experience when signaled (Hektner, et al., 2007).
Specifically, this study employs the Experience Sampling Method (ESM;
Hektner, Schmidt, \& Csikszentmihalyi, 2007) where learners answer short
questions about their experience when signaled. This approach is both
sensitive to changes in engagement over time, as well as between
learners and allows us to understand engagement and how factors impact
it in more nuanced and complex ways (Turner \& Meyer, 2000).Though
time-consuming to carry out, ESM can be a powerful measure that
leverages the benefits of both observational and self-report measures,
allowing for some ecological validity and the use of closed-form
questionnaires amenable to quantitative analysis (Csikszentmihalyi \&
Larson, 1987). Despite the logistic challenge of carrying out ESM in
large studies, some scholars have referred to it as the ``gold
standard'' for understanding individual's subjective experience
(Schwarz, Kahneman, \& Xu, 2009).

Research has shown us how the use of ESM can lead to distinct research
contributions and also suggests how ESM can be put to use for the
purpose of the present study. For example, Shernoff, Csikszentmihalyi,
Schneider, and Shernoff (2003) examined engagement through the use of
measures aligned with flow theory, namely, using measures of
concentration, interest, and enjoyment (Csikszentmihalyi, 1997). In a
study using the same measures of engagement (Shernoff et al. (2016) used
an observational measure of challenge and control (or environmental
complexity) and found that it significantly predicted engagement, as
well as self-esteem, intrinsic motivation, and academic intensity.
Schneider et al. (2016) and Linnansaari et al. (2015) examined features
of optimal learning moments or moments in which students report high
levels of interest, skill, and challenge, as well as their antecedents
and consequences. Similar to ESM in that through its use engagement can
be studied in a more context-sensitive, still other scholars have used
daily diary studies to examine engagement as a function of
autonomy-supportive classroom practices (Patall, Vasquez, Steingut,
Trimble, \& Pituch, 2015; Patall, Steingut, Vasquez, Trimble, \&
Freeman, 2017). This past research that used ESM (or daily diary
studies) to study engagement has shown that ESM can be used to
understand fine-grained differences in learning activities, such as the
aspects of work with data that are the focus of this study.

Other research shows us that there are newer approaches to analyzing ESM
data that can contribute insights into the context-dependent nature of
engagement in a more fine-grained way. For example, Strati et al. (2017)
explored the relations between engagement to measures of teacher
support, finding associations between instrumental support and
engagement and powerfully demonstrating the capacity of ESM to
understand some of the context-dependent nature of engagement.
Similarly, Poysa et al. (2017) used a similar data analytic approach as
Strati et al. (2017), that is, use of crossed effects models for
variation within both students and time points, both within and between
days. These studies establish the value of the use of ESM to understand
the context-dependent nature of engagement and that such an approach may
be able to be used to understand engaging in work with data.
Additionally, these recent studies (particularly the study by Strati and
colleagues) show that how effects at different levels are treated,
namely, how variability at these levels is accounted for through random
effects as part of mixed effects models, is a key practical
consideration for the analysis of ESM data.

One powerful and increasingly widely used way to examine
context-dependent constructs, such as engagement holistically is the use
of \emph{profiles}, or groups of variables that are measured. This
profile approach is especially important given the multidimensional
nature of engagement. Profiles are commonly used as part of what are
described as person-oriented approaches, those used to consider the way
in which psychological constructs are experienced together and at once
in the experiences of learners. In the context of the present study,
this approach can help to identify naturally occurring profiles of
engagement, or engagement as reported by youth via ESM during particular
moments. Note that in the present study, ESM involves asking youth about
to report on their experience at the time they were signaled (rather
than, for example, before or after the program, which traditional
surveys are well-suited for).

In this study, profiles seek to understand how students experience work
with data (through the lens of engagement and its in-the-moment
conditions) in a more holistic way. There are some recent studies taking
a profile approach to the study of engagement (i.e., Salmela-Aro,
Moeller, Schneider, Spicer, \& Lavonen, 2016a; Salmela-Aro, Muotka,
Alho, Hakkarainen, \& Lonka, 2016b; Van Rooij, Jansen, \& van de Grift,
2017; Schmidt, Rosenberg, \& Beymer, 2018), though none have done so to
study youths' engagement in work with data.

The profile approach has an important implication for how we analyze
data collected from ESM about youths' engagement, in particular when we
consider how to understand engagement as a multi-dimensional construct,
and one with momentary, or instructional episode-specific, conditions
(Csikszentmihalyi, 1990). We know from past research that engagement can
be explained in terms of different patterns among its individual
components (Bergman \& Magnusson, 1997), in the present case its
cognitive, behavioral, and affective components. Because learners'
engagement includes cognitive, behavioral, and affective aspects
experienced together at the same time, it can be experienced as a
combined effect that is categorically distinct from the effects of the
individual dimensions of engagement. This combined effect can be
considered as profiles of engagement.

Past studies have considered profiles of cognitive, behavioral, and
affective aspects of engagement. For example, to account for the
context-dependent nature of engagement, some past studies have used
other measures to predict engagement, such as use of in-the-moment
resources and demands (Salmela-Aro et al., 2016b) or, in the case of the
study reviewed in the previous section, use of instructional activities
and choice (Schmidt et al., 2018). Different from this past research,
another potential way to account for the context-dependent nature of
engagement is to consider both engagement and its conditions at once.
Since a profile approach emphasizes the context-dependent nature of
development and the impact of not only external but also
intra-individual factors. As in the present study, youths' perceptions
of challenge and competence, also collected via ESM, are used along with
the measures of engagement to construct profiles of engagement. Thus,
the profiles of engagement include youths' responses to five ESM items
for their cognitive, behavioral, and affective engagement and their
perceptions of how challenging the activity they were doing is and of
how competent at the activity they are.

\section{Need for the Present Study}\label{need-for-the-present-study}

While many scholars have argued that work with data can be understood in
terms of the capabilities learners develop and the outcome learners
achieve, there is a need to better understand learners' (and youths')
experiences working with data. The present study does this through the
use of contemporary engagement theory and innovative methodological and
analytic approaches. Doing this can help us to understand work with data
in terms of learner's experience, which we know from past research
impacts what and how students learn (Sinatra et al., 2015). Knowing more
about students' engagement can help us to design activities and
interventions focused around work with data that are more engaging and
which provide more support to learners in terms of their perceptions of
challenge and their own competence. In addition to this general need to
study engagement in work with data through the lens of engagement, no
research that I am aware of has examined work with data or data analysis
more generally in the context of outside-of-school programs. These
settings are potentially rich with opportunities for highly engaged
youth to analyze authentic data sources. Finally, little research has
examined how data is part of the experiences of youth in
out-of-school-time programs, despite its place as one of a few core
practices in STEM.

\section{Conceptual Framework and Research
Questions}\label{conceptual-framework-and-research-questions}

To summarize, the present study is about how learning activities
involving various aspects of work with data can be understood in terms
of engagement. Its context is out-of-school-time STEM enrichment
programs designed to meet guidelines for best practices. The conceptual
framework in the present study is presented in Figure 2 and is laid out
in the remainder of this section.

\begin{figure}

{\centering \includegraphics[width=0.8\linewidth]{images/figure2} 

}

\caption{A conceptual framework for this study with research questions labeled}\label{fig:unnamed-chunk-2}
\end{figure}

There are five aspects of work with data synthesized from past research
(i.e., Hancock et al., 1992; Lehrer \& Romberg, 1996; Wild \& Pfannkuch,
1999):

\begin{enumerate}
\def\labelenumi{\arabic{enumi}.}
\tightlist
\item
  Asking questions or identifying problems
\item
  Making observations
\item
  Generating data
\item
  Data modeling
\item
  Interpreting and communicating findings
\end{enumerate}

In Figure 2, engagement in work with data is associated with different
profiles of engagement and its conditions, referred to as profiles of
engagement in the remainder of this manuscript. The theoretical
framework for the profile approach suggests that engagement is a
multi-dimensional construct consisting of cognitive, behavioral, and
affective dimensions of engagement and perceptions of challenge and
competence. In addition, a pre-program measure of learners' individual
interest in STEM is hypothesized to be associated with the profiles and
the relations of work with data and the profiles. The ESM responses that
make up the profiles are associated with students, instructional
episodes (or the moments--or segments--for which youth are asked to
respond to the ESM signal), and program effects that must be accounted
for (Strati et al., 2017). These three random effects represent
systematic variation at these different ``levels'', in that the modeling
approach can account for how much variability in the dependent variable
is explained exclusively by the presence of these (student,
instructional episide, and program) levels, or groups, that are present
given the data collection method.

As depicted in the Figure 2, the four research questions are formalized
as follows:

\begin{enumerate}
\def\labelenumi{\arabic{enumi}.}
\tightlist
\item
  What is the frequency and nature of opportunities for youth to engage
  in each of the five aspects of work with data in summer STEM programs?
\item
  What profiles of engagement emerge from data collected via ESM in the
  programs?
\item
  How do the five aspects of work with data relate to profiles of
  engagement?
\item
  How do youth characteristics relate to profiles of engagement?
\end{enumerate}

\chapter{Method}\label{method}

\section{Context}\label{context}

The setting for the present study is nine out-of-school STEM programs
designed around best practices in urban areas in the Northeast United
States during the summer of 2015. These are described in the Appendix
with pseudonyms for the program names. Two intermediary organizations
contracted by the urban area school districts to administer the summer
programs. The two intermediaries were responsible for soliciting and
enrolling youth; establishing guidelines for the design of the programs,
and the goals of the programs; and provide training and professional
development for the program's staff. A key difference between the
intermediary organizations was that one separated academic and
enrichment-related activities, whereas, in another, which was more
closely involved in the day-to-day activities of the program, the
academic and enrichment components were more integrated, which may have
program-specific effects on youths' engagement. Many of the programs aim
to involve youth in work with data. These learning environments bring
together youth activity leaders, educators, and those with technical
expertise in STEM domains. Youth spent around three hours per day for
four days per week for the approximately four-week programs, which were
taught by youth activity leaders and scientists, engineers, and other
community members with technical expertise.

\section{Participants}\label{participants}

Participants consist of 203 youth. Youth in these programs are from
diverse racial and ethnic backgrounds (see Table 1). Most participants
are around 13 years old (from youth whose age was available: \emph{M} =
12.71, \emph{SD} = 1.70, \emph{min.} = 10.75, \emph{max.} = 16.36).
Detailed demographic characteristics of youth are presented in the
table.

\begin{table}

\caption{\label{tab:unnamed-chunk-3}Demographic characteristics of youth}
\centering
\begin{tabular}[t]{lr}
\toprule
Youth & Percentage\\
\midrule
Sex & \\
Male & 50\\
Female & 50\\
Race/Ethnicity & \\
Hispanic & 48\\
\addlinespace
White & 6\\
Black & 36\\
Multi-racial & 3\\
Asian/Pacific Islander & 7\\
Parent Education & \\
\addlinespace
High School or Below & 79\\
Graduated from College (B.A. or B.S.) & 21\\
\bottomrule
\end{tabular}
\end{table}

\section{Procedure}\label{procedure}

Youth completed a pre-survey before the program including questions
about their experience in STEM, intention to pursue a STEM major or
career, and questions for other motivation and engagement-related
measures. At the beginning of the programs, youth were introduced to the
study and the phones used for data collection related to the ESM. As
indicated in the earlier section, ESM is a method of data collection
that involves asking (signalling) youth to respond to short questions on
phones that they were provided. Youth are signaled at random times
(within intervals, so that the signals were not too near or far apart)
in order to obtain a sample of youths' experiences throughout the
program. ESM data were collected two days each week, for three weeks
(weeks 2-4 of the program). In all of the programs, about equal
video-recording time was dedicated to classroom and field experiences.
This detail is important because programs associated with one of the
intermediaries rotated between classroom and field experience days,
while the other used the first half of each day for one (i.e., classroom
activities) or the other (i.e., field experience days).

Each day, youth were signaled four times. These signals were at the same
time for all of the youth within their program, but at different times
between programs and between days within programs (with the constraint
that no two signals could occur less than ten minutes apart). All of the
programs were video-recorded by research team members. So that measures
corresponding to the video and ESM data can be matched, videos include a
signal from the video-recorder identifying the ESM signal to which youth
responded at that point in the video.

\section{Data Sources and Measures}\label{data-sources-and-measures}

Data sources consist of self-reported ESM measures of engagement and
youths' perceptions of themselves and the activity, pre-survey measures
of youths' interest, youths' demographic information, and
video-recordings of programs.

\subsection{ESM measures of youths' engagement and its conditions for
the
profiles}\label{esm-measures-of-youths-engagement-and-its-conditions-for-the-profiles}

Measures for engagement and its conditions were constructed from three
ESM responses for engagement and two ESM responses for the conditions of
engagement. The three variables for engagement are for learning (for the
cognitive engagement construct), working hard (for behavioral
engagement), and enjoying (for affective engagement). The variables for
the conditions are for perceived challenge and perceived competence. All
five items are used to construct profiles. Each of the ESM items
consisted of the item text and the following four item response options,
of which youth were directed to select one: Not at all (associated with
the number 1 on the survey), A little (2), Somewhat (3), and Very Much
(4), as presented in Table 2.

\begin{table}

\caption{\label{tab:unnamed-chunk-4}ESM measures for profiles}
\centering
\resizebox{\linewidth}{!}{\begin{tabular}[t]{ll}
\toprule
Construct & Item\\
\midrule
Cognitive engagement & As you were signaled, were you learning anything or getting better at something?\\
Behavioral engagement & As you were signaled, how hard were you working?\\
Affective engagement & As you were signaled, did you enjoy what you are doing?\\
Perceived challenge & As you were signaled, how challenging was the main activity?\\
Perceived competence & As you were signaled, were you good at the main activity?\\
\bottomrule
\end{tabular}}
\end{table}

\subsection{Survey measures of
pre-interest}\label{survey-measures-of-pre-interest}

Measures of youths' pre-interest are used as youth-level influencers of
the profiles. In particular, three items adapted from Vandell, Hall,
O'Cadiz, and Karsh (2012) were used, with directions for youth to rate
their agreement with the items' text using the same scale as the ESM
items: Not at all (associated with the number 1 on the survey), A little
(2), Somewhat (3), and Very Much (4). The measure was constructed by
taking the maximum value for the scales for the different content areas
(science, mathematics, and engineering), so that the value for a youth
whose response for the science scale was 2.5 and for the mathematics
scale was 2.75 would be 2.5. The items are presented in Table 3.

\begin{table}

\caption{\label{tab:unnamed-chunk-5}Measure for pre-program interest in STEM}
\centering
\begin{tabular}[t]{ll}
\toprule
Construct & Items.text\\
\midrule
Individual interest in STEM & I am interested in science / mathematics / engineering.\\
 & At school, science / mathematics / engineering is fun\\
 & I have always been fascinated by science / mathematics / engineering)\\
\bottomrule
\end{tabular}
\end{table}

\subsection{Codes from video-recordings for work with
data}\label{codes-from-video-recordings-for-work-with-data}

Different aspects of work with data are identified from
video-recordings. Specifically, codes for work with data were generated
on the basis of the activity that the youth activity leaders were
facilitating. The activity youth activity leaders were facilitating were
from the STEM-Program Quality Assessment (STEM-PQA; Forum for Youth
Investment, 2012), an assessment of quality programming in after school
programs. I then identified the specific activities that corresponded to
the five aspects of work with data, as defined in Table 4. Details on
how the measure aligns with the original STEM-PQA on which this measure
is based are presented in the appendix. Note that these codes were
unique to each signal to which youth responded (but were not unique to
each youth, as youth in the same program were signaled at the same
time). I will discuss limitations to use of the STEM-PQA for work with
data in the discussion.

\begin{table}

\caption{\label{tab:unnamed-chunk-6}Coding Frame for Work With Data}
\centering
\resizebox{\linewidth}{!}{\begin{tabular}[t]{ll}
\toprule
Code & Description\\
\midrule
Asking questions & Discussing and exploring topics to investigate and pose questions.\\
Making observations & Watching and noticing what is happening with respect to the phenomena or problem being investigated.\\
Generating data & Figuring out how or why to inscribe an observation as data and generating coding frames or measurement tools.\\
 & \\
Data modeling & Understanding and explaining phenomena using models of the data that account for variability or uncertainty.\\
Interpreting and communicating findings & Discussing and sharing and presenting findings.\\
\bottomrule
\end{tabular}}
\end{table}

In February, 2017, raters contracted by American Institute of Research
(AIR) were trained in the use of the Program Quality Assessment tool
(PQA)--the broader assessment tool for which the STEM-PQA is a
supplement. Raters completed a four-hour online training module on the
overall PQA tool and then attended an in-person two-day training led by
a trainer from the David P. Weikart Center for Youth Program Quality,
the tool's publisher, where they learned about the instrument, trained
on its use, and then established inter-rater reliability with a master
coder.

For the STEM-PQA, three of the same raters contracted by AIR to code the
(overall) PQA measure used the STEM-PQA supplement to score one video
segment, for which there were no disagreements on scoring for any of the
items. The programs were divided up among all of the raters, so raters
coded some of the videos for all of the programs. When the raters
encountered a situation that was difficult to score, they would all
discuss the issue by telephone or more often by email after viewing the
video in question and reach a consensus on how to score the specific
item.

\subsection{Demographic variables
used}\label{demographic-variables-used}

In addition to the measures described in this section, demographic
information for youths' gender and their racial and ethnic group are
used to construct demographic variables for gender and membership in an
under-represented (in STEM) group; membership in an under-represented
group are identified on the basis of youths' racial and ethnic group
being Hispanic, African American, Asian or Pacific Islanders, or native
American.

\section{Data Analysis}\label{data-analysis}

The steps for both preliminary and the primary analyses are described in
this section.

\subsection{Preliminary analyses}\label{preliminary-analyses}

First-order Pearson correlations and the frequency, range, mean, and
standard deviations are first examined for all variables. In addition,
the frequency of the codes for aspects of work with data, and the
numbers of responses by youth, program, and instructional episode are
examined.

\subsection{Analysis for Research Question \#1 (on the frequency and
nature of work with
data)}\label{analysis-for-research-question-1-on-the-frequency-and-nature-of-work-with-data}

There are two primary steps taken to answer this question, one more
quantitative in nature and one more qualitative.

Specifically, first, the frequency of the codes for the individual
aspects of work with data from the STEM PQA measure of work with data
(described above in the measures) are calculated. Note that this coding
frame focused on the degree of instructional support the activity
leaders provided for youth to work with data.

However, this framework provided no context on how particular types of
work with data were enacted. Qualitative differences in \emph{how}, for
example, youth were asking questions are not evident from the STEM-PQA
codes used to find the frequencies of the aspects of work with data. In
order, to provide more context in the description of how work with data
in the context of summer STEM programs, all of the segments were coded
using an open-ended, qualitative approach. Three research assistants
were trained for approximately eight hours over four meetings. Then,
each research assistant coded all of the segments associated with one of
the videos. Two coders coded every segment, except for the 77 (out of
the total 248) segments that the STEM-PQA coding that indicated no
aspects of work with data were present. For these 77 segments, only one
coder coded each segment.

The coders used the following five guiding questions, assocatiated with
each of the five aspects of work with data, for the qualitative coding::

\begin{itemize}
\tightlist
\item
  When asking questions or defining problems is coded, what, if any are
  the questions or problems? Who is asking the question (i.e teacher or
  student)
\item
  When making observations is coded, what are youth doing?
\item
  When generating data is coded, how, if they are, are youth collecting
  or recording data?
\item
  When analyzing or modeling data is coded, what analysis are they
  doing, or what models are they using? Are they talking about
  variability or uncertainty? If so, how?
\item
  When interpreting and communicating findings is coded, what are youth
  interpreting or how are they communicating?
\end{itemize}

This coding took around 75 hours of coding by the research assistants.
After coding all of the segments for each program, the coders and I met
to discuss potential issues that emerged throughout the coding, and to
clarify how they applied the coding frame (so the coders and I met nine
times during the process to discuss the coding). I then read through all
of the codes for all of the segments then made notes associated with
each of the five aspects of work with data. I used these notes to write
detailed descriptions of each of the aspects of work with data, which I
grouped into the themes. I present these themes in the section of the
results for this question.

\subsection{Analysis for Research Question \#2 (what profiles of
engagement
emerge)}\label{analysis-for-research-question-2-what-profiles-of-engagement-emerge}

To answer this question, Latent Profile Analysis (LPA; Harring \& Hodis,
2016; Muthen, 2004) is used. LPA allows for capturing the
multidimensional nature of engagement through profiles. A key benefit of
the use of LPA, in addition to likelihood estimation-based fit indices,
is probabilities of an observation being a member of a cluster (unlike
in cluster analysis). These profiles make it possible to analyze the
multivariate data collected on engagement in a way that balances the
parsimony of a single model.

For these models, five variables were included: the three measures of
engagement (cognitive, behavioral, and affective) and the two conditions
for engagement (perceptions of challenge and competence). In addition,
solutions with between two and 10 profiles were considered. As part of
LPA, model type selection--in terms of which model type is used and how
many profiles are identified--is a key topic. For the present study, six
models were considered:

\begin{enumerate}
\def\labelenumi{\arabic{enumi}.}
\tightlist
\item
  Varying means, equal variances, and covariances fixed to 0 (model
  \emph{one} type)
\item
  Varying means, equal variances, and equal covariances (model
  \emph{two} type)
\item
  Varying means, varying variances, and covariances fixed to 0 (model
  \emph{three} type)
\item
  Varying means, varying variances, and equal covariances (model
  \emph{four} type)
\item
  Varying means, equal variances, and varying covariances (model
  \emph{five} type)
\item
  Varying means, varying variances, and varying covariances (model
  \emph{six} type)
\end{enumerate}

The MPlus software (Muthen \& Muthen, 1998-2017) is used to carry out
LPA as part of this study. In order to more flexibly carry out LPA, an
open-source tool, tidyLPA (Rosenberg, Schmidt, \& Beymer, 2018), was
developed. This tool provides interfaces to both the MPlus software and
to the open-source mclust software. In addition to being used as part of
this study, this package is provided free of use to other analysts as
the first tool dedicated to carrying out LPA as part of the R software.
More details on the statistical software developed and included in the
Appendix.

To select a solution in terms of the model type and the number of
profiles to be interpreted and used in subsequent analyses, a number of
fit statistics and other considerations are taken into account. These
include a range of information criteria (AIC, BIC, sample adjusted BIC
{[}SABIC{]}, consistent AIC {[}CAIC{]}), statistics about the quality of
the profile assignments (entropy, which represents the mean posterior
probability), statistical tests (Vu-Lo-Mendell-Rubin LRT {[}VLMR{]},
Lo-Mendell-Rubin LRT {[}LMR{]}, and the bootstrapped LRT {[}BLRT{]}),
and concerns of interpretability and parsimony are used. As described in
more detail in the section of the results pertaining to this question,
on the basis of these criteria, the \emph{model one type, six profiles}
solution is selected and used as part of subsequent analyses.

\subsection{Analysis for Research Question \#3 (how work with data
relates to
engagement)}\label{analysis-for-research-question-3-how-work-with-data-relates-to-engagement}

Broadly, this question is focused on how work with data as coded from
video-recordings of the programs, relates to the profiles. For the
primary results for this question, mixed effects models that account for
the cross-classification of the instructional episode (because of the
dependencies of the responses associated with each of the 248 distinct
ESM signals) and youth are used and for the ``nesting'' of both within
each of the nine programs are used. The \emph{lme4} R package (Bates,
Martin, Bolker, \& Walker, 2015) is used. All of the models for this and
the subsequent research question use random effects for youth,
instructional episode, and program effects. Youth and instructional
episode can be considered to be crossed with both nested within the
program.

The probability of a response belonging to the profile is the dependent
variable and the aspects of work with data are the independent variable.
There are six models, for each of the six profiles. Because the outcome
from LPA is not a hard classification (i.e., an observation is in a
profile---or not) but a probability, the dependent variable is treated
as a continuous variable.

First, null models with only the random parts (i.e., random youth,
instructional episode, and program effects) are specified. Then, the
five aspects of work with data are added as predictors to the model. The
results will be interpreted on the basis of which of the statistical
significance and the magnitude and direction of the coefficients
associated with these five predictors. For example, if the coefficient
for the effect of the asking questions aspect of work with data upon one
of the profiles is 0.10, and is determined to be statistically
significant, then this would indicate that when youth are engaged in
this aspect of work with data, then they are ten percentage points more
likely to report a response in a particular profile.

Because the results were found to be identical when the aspects of work
with data and the youth characteristics are considered in separate and
in the same model, the results from the two sets of variables being in
the same model are used for both to provide answers to both this and the
next research question. Note that a composite for work with data (made
as the sum of the individual aspects of work with data) was considered,
but as it did only yielded one (small) statistically significant result,
the results for this analysis are not presented in the results.

\subsection{Analysis for Research Question \#4 (how youth
characteristics relate to
engagement)}\label{analysis-for-research-question-4-how-youth-characteristics-relate-to-engagement}

Research question \#4 is focused on how the relationships of work with
data differ on the basis of youth characteristics--their pre-program
interest, gender and URM status. Like for the previous research
question, models that account for the cross-classification of the
instructional episodes and the youth are used. The dependent variable is
again the probability of a response being in the profile. The three
youth characteristics (pre-program interest in STEM, gender (entered s a
dummy code with the value of ``1'' indicating female), and URM status
(also entered as a dummy code, with ``1'' indicating a youth from a URM
group) are added as predictors. Like for the previous research question,
the statistical significance and the magnitude and direction of the
coefficients associated with each predictor are interpreted to answer
this question. For example, and similar to the interpretation of the
predictors associated with RQ \#3, if the relationship between
pre-program interest and a profile is 0.05, then for each one-unit
increase in pre-program interest, then youth are are five percentage
points more likely to report a response in a particular profile.

As described in the previous sub-section, because the results were very
similar when the aspects of work with data and the youth characteristics
were added in \emph{separate} models compared to when they were included
in the same model, the results for both sets of predictors in the same
model are presented and interpreted. In addition, interactions between
statistically significant aspects of work with data and all of the youth
characteristics are examined, though because none of these interactions
were found to be statistically significant, they are not included with
the results.

\section{Sensitivity Analysis}\label{sensitivity-analysis}

For observational studies, such as the present study, it can be
important to determine how robust an inference is to alternative
explanations. One approach to addressing this is sensitivity analysis,
which involves quantifying the amount of bias that would be needed to
invalidate an inference. Using the approach described in Frank,
Maroulis, Duong, and Kelcey (2013), I carried out sensitivity analysis
for inferences made relative to key findings. I used the R package
konfound (Rosenberg, Xu, \& Frank, 2018). The result, and what is used
to interpret and contextualize findings, is a numeric value for each
effect that indicates the proportion of the estimate that would have to
be biased in order to invalidate the inference. I use these to interpret
and contextualize the statistically significant findings. Higher values
indicate more robust estimates in that the inferences would still hold
even if there were substantial bias in the estimate and that are
interpreted as robust findings, while lower values, when present,
indicate less robust findings that I interpret with more caution.

\chapter{Results}\label{results}

In this section, results associated with the preliminary analysis and
the four research questions are presented.

\section{Preliminary results}\label{preliminary-results}

\subsection{Descriptive statistics for the study
variables}\label{descriptive-statistics-for-the-study-variables}

First, descriptive statistics for all of the study variables (overall
pre-interest and the five variables that are used to estimate the
profiles) are presented in Table 5. These values suggest moderately high
levels of the three dimensions of engagement (with mean values between
2.768 (\emph{SD} = 1.063) for cognitive engagement, and 2.863 (\emph{SD}
= 1.044), for behavioral engagement, on one-four scales), high
perceptions of competence (\emph{M} = 3.000 (\emph{SD} = 0.952)) and
pre-program interest (\emph{M} = 3.044 (\emph{SD} = 0.901); the only
continuous variable that was not measured using ESM) and lower
perceptions of challenge (\emph{M} = 2.270 (\emph{SD} = 1.117)).

\begin{table}

\caption{\label{tab:unnamed-chunk-7}Descriptive statistics for study variables}
\centering
\begin{tabular}[t]{lrrr}
\toprule
 & n & Mean & SD\\
\midrule
Pre-interest & 181 & 3.044 & 0.901\\
Cog. eng. & 2969 & 2.768 & 1.063\\
Beh. eng. & 2959 & 2.863 & 1.044\\
Aff. eng. & 2970 & 2.831 & 1.051\\
Challenge & 2970 & 2.270 & 1.117\\
Competence & 2970 & 3.000 & 0.952\\
\bottomrule
\end{tabular}
\end{table}

\subsection{Correlations among the study
variables}\label{correlations-among-the-study-variables}

Next, correlations between the variables that are used to create the
profiles are presented in Table 6. These correlations among the
variables used to construct the profiles, which range from \emph{r} =
.08 through \emph{r} = .60 (all statistically significant), show
moderate relations. Pre-interest was correlated with the variables used
to construct the profiles to a small degree (with \emph{r} values
ranging between .06 and .14).

\begin{landscape}\begin{table}

\caption{\label{tab:unnamed-chunk-8}Correlations among study variables}
\centering
\begin{tabular}[t]{lllllll}
\toprule
 & Pre-interest & Cog. eng. & Beh. eng. & Aff. eng. & Challenge & Competence\\
\midrule
Pre-interest &  &  &  &  &  & \\
Cog. eng. & .14 &  &  &  &  & \\
Beh. eng. & .13 & .60 &  &  &  & \\
Aff. eng. & .12 & .59 & .57 &  &  & \\
Challenge & .15 & .30 & .27 & .27 &  & \\
Competence & .06 & .40 & .41 & .47 & .08 & \\
\bottomrule
\end{tabular}
\end{table}
\end{landscape}

\section{Results for Research Question
\#1}\label{results-for-research-question-1}

\subsection{Frequency of work with
data}\label{frequency-of-work-with-data}

Out of the 248 instructional episodes, 236 were code-able for work with
data; for the 12 that were not codeable, issues with the
video-recordings were the primary source of the missing data. Of the 236
code-able instructional episodes, 170 (72\%) were coded as involving
\emph{any} of the aspects of work with data. Table 7 includes the
frequency of the specific aspects of work with data, with interpreting
and communicating findings being the most present (occurring in 47\% of
the coded instructional episodes), followed by generating data (in 45\%
of the instructional episodes), asking questions (in 39\%), data
modeling (29\%), and then making observations (26\%).

Note that these results are for codes applied to approximately
ten-minute (video-recorded) instructional episodes and that the aspects
of work with data could co-occur. On average, there were 1.86 (\emph{SD}
= 1.61) aspects of work with data present in each 10-minute
instructional episode. This indicates that, on average, youth were
engaged in around two of aspects of the work with data during each
instructional episode. There was considerable variation in the extent to
which these types of work with data were supported in each program (see
the Appendix).

\subsection{Nature of work with data}\label{nature-of-work-with-data}

For these results, the different aspects of work with data were looked
at in more detail using an open-ended, qualitative approach in order to
better understand the nuance and the specific nature of what was going
on during these episodes in terms of how students work with data. This
coding, which showed there to be distinct, qualitative differences in
the particular ways youth worked with data when, for example, they were
data modeling, resulted in approximately three to four sentence notes
from each of two raters for every instructional episode and showed the
specific nature of work with data.

\subsubsection{Asking questions or identifying
problems}\label{asking-questions-or-identifying-problems}

Among the instructional episodes focused on asking questions (as
determined through coding with the STEM-PQA measure), qualitative
analysis revealed that around one-third were focused on asking questions
focused on youth working to understand the phenomenon or problem they
were investigating. For example, in an instructional episode during the
\emph{Ecosphere} program in which youth constructed inclined tables to
study how water moved throughout the ecosystem, the youth activity
leader prompted youth to generate hypotheses of what would happen when
water was poured onto the table, before pouring the water. Other
instructional episodes not focused on understanding the phenomenon or
specific problem showed that many instructional episodes were focused
around predicting, conjecturing, or hypothesizing. In these cases, the
code was applied to instances in which the youth were asking generic
questions (i.e., about how they do an assignment) or when the instructor
was asking youth questions (i.e., math-related questions). For example,
in the \emph{Marine Investigators} program, youth visited a water
treatment site, and were provided opportunities to ask questions about
what they saw.

\subsubsection{Making observations}\label{making-observations}

In the instructional episodes during which the STEM-PQA revealed that
youth were making observations, the vast majority of these were focused
on observing phenomenon in the field, or, in the case of
engineering-focused programs, noticing what was going on with a
particular design. For example, in the \emph{Building Mania} program,
youth constructed Rube Goldberg machines; youth were prompted by the
activity leaders to notice how changes in their design led to
differences in how far objects were launched or rolled. In just a few
cases, making observations was focused on making observations not of
phenomena, but of the instructor. For example, in the \emph{Adventures
in Mathematics} program, instances in which youth observed other youth
or the youth activity leader solving a mathematics problem was often
coded as involving making observations.

\subsubsection{Generating data}\label{generating-data}

Around half of these episodes, for youth generating sources of data (as
indicated by the STEM-PQA), the instructional episodes were focused
around writing down observations made of a phenomenon, recording
information from experiments, or recording the results of a trial (in
engineering contexts). For example, in the \emph{Marine Investigators}
program, youth collected pieces of recyclable plastic, bringing them
back to the classroom and counting them for each location they were
collected. The other half of the cases were most often instructional
episodes in which youth were writing down what the youth activity leader
was saying or were focused on collecting specimens (but not writing them
down) entering them into a spreadsheet, or otherwise recording them as
data. For example, again in the \emph{Marine Investigators} program,
youth used nets to collect saltwater organisms, which they then
transported in buckets back to the classroom setting for subsequent
analysis. While these specimens could be considered as data, at least in
the instructional episode described, youth did not inscribe notes or any
other observations on the specimens they were collecting, and so data
was not generated (at this stage).

\subsubsection{Data modeling}\label{data-modeling}

A large majority of the instructional episodes coded (with the STEM-PQA)
for data modeling were focused on youths' uses of statistical and
mathematical models. For example, in the \emph{Comunidad de Aprendizaje}
program, youth accessed nationally-representative data and were tasked
to solve problems, like finding out what percentage of people engage in
particular activities, like donating to charity. In a small number of
instructional episodes, this aspect of work with data was present when
the youth activity leader, rather than students, was doing the modeling,
or the model was not one that could generate data. For example, in the
\emph{Marine Investigators} program, a youth activity leader used a
plush toy seal designed to teach youth about anatomy and the dangers of
aquatic mammals consuming trash and recyclables.

\subsubsection{Interpreting and communicating
findings}\label{interpreting-and-communicating-findings}

In around half of the instructional episodes in which youth were
interpreting and communicating findings (as coded by the STEM-PQA),
youth were sharing what they found from an investigation or the results
of using the product they designed. For example, in the \emph{Comunidad
de Aprendizaje} program, youth participated in an activity designed to
support their thinking about creating a product to bring to market; the
youth activity leaders described this as being akin to the television
show the \emph{Shark Tank}. In one instructional episode, the youth
activity leader asks youth to think of an idea that would make an
investor willing to invest in; students shared their ideas, describing
what their ideas was, why it was a good idea, how much they could sell
it for, and what their profit would be, while fielding questions from
youth activity leaders and their peers. Interpreting and communicating
findings was also commonly present in instructional episodes in which
youth were debating the findings of an investigation, such as the
results of calculations for the amount of recyclables entering waterways
(in \emph{Marine Investigators}). In the other half of the responses,
youth were most commonly communicating about topics other than the
results of an investigation or design process, such as trying to find
out the answer to a question posed by the youth activity leader, or the
youth activity leader was who was doing the interpreting and
communicating. For example, in the \emph{Adventures in Mathematics}
program, the youth activity leader helped youth to solve problems on a
worksheet, asking guiding questions to help youth start to solve
problems on their own.

\section{Results for Research Question \#2: What profiles of youth
engagement and its conditions emerge from experiential data collected in
the
programs?}\label{results-for-research-question-2-what-profiles-of-youth-engagement-and-its-conditions-emerge-from-experiential-data-collected-in-the-programs}

On the basis of the selection criteria, the six profile solution with
varying means, equal variances and covariances fixed to 0 emerged as the
best fit of the data. This was on the basis of fit statistics,
statistical tests, and concerns of interpretability and parsimony. The
model demonstrated superior fit on the basis of the information criteria
(AIC and BIC) and on the basis of the measure of classification accuracy
(entropy). A seven profile solution with the same specifications
regarding means, variances and covariances was also a similarly good fit
(and is presented in the Appendix), but the 6 profile solution was
ultimately chosen on the basis of parsimony and interpretability. For
the selected model, presented below in Figure 3, the raw data and the
data that are centered to have a mean equal to 0 and a standard
deviation of 1 (thus, the y-axis on each of the plots is labeled
``Z-score'').

\begin{center}\includegraphics[width=1\linewidth]{rosenberg-dissertation_files/figure-latex/unnamed-chunk-10-1} \end{center}

\begin{center}\includegraphics[width=1\linewidth]{rosenberg-dissertation_files/figure-latex/unnamed-chunk-10-2} \end{center}

This solution is characterized by:

\begin{itemize}
\tightlist
\item
  A \emph{universally low} profile, characterized by low levels of
  engagement and challenge and competence
\item
  An \emph{only behaviorally engaged} profile, with moderate levels of
  behavioral engagement, very low affective engagement, and moderately
  (low) levels of cognitive engagement and challenge and competence
\item
  An \emph{only affectively engaged} profile, with moderate levels of
  affective engagement, low levels of behavioral engagement, and
  moderately (low) levels of cognitive engagement and challenge and
  competence
\item
  A \emph{all moderate} profile, with moderate levels of the three
  dimensions of engagement and challenge and competence
\item
  An \emph{engaged and competent but not challenged} profile,
  characterized by high levels of each of the three dimensions of
  engagement and of competence, but with low levels of challenge
\item
  A \emph{full} profile, with high levels of engagement, challenge, and
  competence
\end{itemize}

The number of observations associated with each of the profiles is
somewhat balanced, with the universally low profile with the largest
number of observations (\emph{n} = 667), followed by the all moderate
profile (\emph{n} = 638). Each of the other four profiles were
associated with 300 to 400 observations.

A relatively simple model (model \emph{one} type; with varying means,
equal variances, and covariances fixed to 0) with six profiles was
selected for use in subsequent analyses. This model has profiles
characterized by both varying levels on both the dimensions of
engagement--cognitive, behavioral, and affective--and youths'
perceptions of challenge and competence. In addition, the number of
observations across the profiles is relatively balanced.

\subsection{Sources of variability in profiles of
engagement}\label{sources-of-variability-in-profiles-of-engagement}

After identifying the \emph{model one type, six profile solution},
sources of variability in these profiles can be explored. These sources
are useful as additional information in their own right for interpreting
the profiles and in order to anticipate the effects of predictor
variables at the youth, instructional episode, and program levels.

First, the proportion of the variability at each of these levels is
explored through the use of null, or variance components, in Table 8.
Again, these are models that only include grouping (i.e., the variable
identifying which youth a response is from, what signal the response is
associated with, and from which program the youth and signal were from)
factors. These models provide insight into at which of these ``levels''
predictors may be able to explain the outcome. For all six profiles, the
ICCs at the program level were very small, from 0.00 to 0.023. This
suggests that very little variability can be explained simply by the
program. For the instructional episode level, the ICCs were also very
small, ranging from 0.004 to 0.011. Finally, the youth-level ICCs ranged
from .099 to .427.

\begin{landscape}\begin{table}

\caption{\label{tab:unnamed-chunk-11}Intra-class correlation (ICC) values for each of the three levels}
\centering
\begin{tabular}[t]{rrr}
\toprule
Instructional Episode & Youth & Program\\
\midrule
0.006 & 0.093 & 0.009\\
0.006 & 0.267 & 0.023\\
0.015 & 0.310 & 0.000\\
0.009 & 0.100 & 0.000\\
0.004 & 0.262 & 0.003\\
0.031 & 0.432 & 0.019\\
\bottomrule
\end{tabular}
\end{table}
\end{landscape}

Looking across these values, considering variability at the program,
instructional episode, and youth levels, most of the explained
variability in the responses is associated with youth; the program and
instructional episode levels were associated with very small values,
suggesting that variables at these levels have minimal variability to
explain. In turn, this suggests that these variables, including those
for work with data, may not have strong effects in terms of their
relations with the profiles.

In terms of specific ICCs at the youth level, the value for the
youth-level ICC was highest for the \emph{full} profile, suggesting that
some youth have a strong tendency to be fully engaged (possibly due to
their initial interest or other individual characteristics and
differences). The other profile characterized by a consistent pattern
across all of the variables--the \emph{universally low} profile--had a
modest ICC, .265. Finally, a large amount of variability is associated
with the residual (variance that is not associated with the program,
instructional episode, or youth levels). This suggests that there is
wide variation in students' responses that may not be readily explained
or predicted.

Variability in terms of the number (and proportion) of profiles each
youth reports can also be considered. These show that there is
substantial variability between youth, in that, when youth, for example,
report \emph{Full} engagement to a greater extent than any other profile
of engagement, these youth (on average) report this engagement in just
over 60\% of their responses. Their other responses were (on average)
associated with a mixture of other profiles. Youth who report more
\emph{Full} engagement than any other profile of engagement are the most
consistent in reporting one of the profiles of engagement, with youth
reporting engagement associated with the \emph{All moderate} profile
doing so just less than 40\% of the time (with other profiles being
associated with the remainder of their responses).

\section{Results for Research Question \#3: Aspects of work with data
and
engagement}\label{results-for-research-question-3-aspects-of-work-with-data-and-engagement}

For this question, models with the aspects of work with data both
separate from and together with the youth characteristics were fit. The
models only with the aspects of work with data yielded very similar
results (see the Appendix for more detail). The models with both
together were also used as part of research question \#4, though they
are presented here (and interpreted in the sections for both results).
In specific, mixed effects models, predicting the probability of
membership in each of the six profiles as the dependent variable--using
the work with data codes as predictors--were specified.

For each model, results, presented in Table 9, use each row to represent
the output from one of the six different models. For example, the first
row includes the results for the model with the probability of a
response being associated with the \emph{Only behavioral} profile as the
dependent variable. The cells across the columns contain the
coefficients (and their standard errors and (\emph{p}-values)) for each
of the predictor variables. Only the coefficients associated with the
aspects of work with data are interpreted to provide results for this
research question, as the youth characteristics are interpreted for
research question \#4 (as described earlier, the results were
practically the same for these sets of predictors being included in
either separate or the same model, so they are included in the same
model). See the appendix for the results from the model with the aspects
of work with data included as predictors in separate (without the youth
characteristics) models. The only relations that were statistically
significant were for the relations between modeling data and the
\emph{full} profile (\(\beta\) = 0.034 (0.017), \emph{p} = .020) and
between generating data and the \emph{full} profile (\(\beta\) = 0.027
(0.015), \emph{p} = .033): When youth were either modeling or generating
data, they were more likely to be fully engaged.

Sensitivity analysis for these effects showed that the effect of
modeling data on \emph{full} engagement was more robust than that for
generating data (also upon \emph{full} engagement): 9.835\% of the
effect of modeling would have to be due to bias to invalidate the
inference about its effect, whereas only 1.884\% of the effect of
generating data would need to be due to bias to invalidate the inference
about its effect. Further explanations and investigations of these
effects are the focus on research question \#4 (in terms of the effect
of youth characteristics) and are discussed in the next chapter.

\begin{landscape}\begin{table}

\caption{\label{tab:unnamed-chunk-12}Results of mixed effects models with the interactions between interest and other characactistics and the composite for work with data}
\centering
\resizebox{\linewidth}{!}{\begin{tabular}[t]{llllllllll}
\toprule
Profile & Intercept & Pre-interest & Gender-Female & URM status & Asking & Observing & Generating & Modeling & Communicating\\
\midrule
Universally low & 0.356 (0.086) (p < .001) & -0.047 (0.022) (p = 0.982) & 0.06 (0.037) (p = 0.051) & -0.01 (0.052) (p = 0.575) & -0.015 (0.018) (p = 0.789) & 0.003 (0.018) (p = 0.427) & -0.014 (0.017) (p = 0.789) & 0.004 (0.019) (p = 0.407) & 0.002 (0.018) (p = 0.461)\\
Only behavioral & 0.107 (0.045) (p = 0.01) & -0.013 (0.012) (p = 0.873) & 0.019 (0.019) (p = 0.159) & 0.031 (0.026) (p = 0.122) & 0.015 (0.015) (p = 0.158) & 0.013 (0.015) (p = 0.191) & 0.014 (0.014) (p = 0.17) & -0.023 (0.016) (p = 0.929) & 0.018 (0.015) (p = 0.115)\\
Engaged and competent but not challenged & 0.022 (0.063) (p = 0.362) & 0.039 (0.016) (p = 0.009) & 0.025 (0.028) (p = 0.188) & -0.012 (0.04) (p = 0.614) & -0.011 (0.015) (p = 0.763) & 0.009 (0.015) (p = 0.266) & -0.014 (0.014) (p = 0.833) & 0 (0.015) (p = 0.504) & 0.004 (0.015) (p = 0.404)\\
All moderate & 0.354 (0.075) (p < .001) & -0.012 (0.019) (p = 0.743) & -0.038 (0.033) (p = 0.875) & -0.076 (0.046) (p = 0.95) & 0.023 (0.017) (p = 0.09) & 0.007 (0.017) (p = 0.342) & 0.012 (0.016) (p = 0.233) & -0.004 (0.018) (p = 0.593) & -0.011 (0.017) (p = 0.747)\\
Only affective & 0.09 (0.04) (p = 0.015) & 0.007 (0.01) (p = 0.264) & -0.02 (0.018) (p = 0.867) & 0.018 (0.025) (p = 0.234) & 0.004 (0.014) (p = 0.397) & -0.017 (0.014) (p = 0.886) & -0.02 (0.013) (p = 0.938) & -0.012 (0.015) (p = 0.8) & 0.016 (0.014) (p = 0.124)\\
Full & 0.094 (0.083) (p = 0.132) & 0.018 (0.021) (p = 0.195) & -0.035 (0.037) (p = 0.827) & 0.043 (0.053) (p = 0.207) & -0.019 (0.016) (p = 0.887) & -0.025 (0.016) (p = 0.94) & 0.027 (0.015) (p = 0.033) & 0.034 (0.017) (p = 0.02) & -0.027 (0.016) (p = 0.956)\\
\bottomrule
\end{tabular}}
\end{table}
\end{landscape}

\section{Results for Research Question \#4: Youth characteristics and
engagement}\label{results-for-research-question-4-youth-characteristics-and-engagement}

For this question, models with the youth characteristics separate from
and together with the aspects of work with data were fit. Like for the
results for the previous question, the models only with the youth
characteristics yielded very similar results; see the Appendix for the
results from the model with the youth characteristics included as
predictors in separate (without the aspects of work with data) models.
Thus, the models presented in the previous section with both youth
characteristics and the aspects of work (see the table above) with data
are interpreted here.

Like for the results for research question \#3, each row is associated
with the results for a single model. For example, the first row is again
associated with the results for the model predicting the probability of
the \emph{Only behavioral} profile, with the cells across the columns
containing the coefficients, their standard errors, and their
\emph{p}-values. These results show that overall pre-interest is
associated with the \emph{Engaged and competent but not challenged}
profile (\(\beta\) = 0.039 (0.016), p = .009). For this effect, 17.879\%
would be needed to invalidate the inference, suggesting a moderately
robust effect. The effect of being a female was not statistically
significant but has a relation of 0.060 (0.037, p = .051) upon the
probability of a response being associated with the \emph{Universally
low} profile. For the effect of gender upon the \emph{Universally low}
profile, 17.843\% of the bias would need to be removed (or the effect
would need to be larger by this percentage) to sustain the inference.

These few, small findings are more surprising than the similarly minimal
relations observed for work with data: as the null models indicate,
there were large ICCs (a large proportion of the variability in the
outcome variables) at the youth-level (as pre-interest, gender, and URM
status are variables associated with this level). However it appears
that the youth level variables of interest to this study were not
effective at explaining much of this variability. This is discussed
further in the next chapter.

\chapter{Discussion}\label{discussion}

\section{Key Findings}\label{key-findings}

\subsection{Key findings for research question \#1 (on the frequency and
nature of work with
data)}\label{key-findings-for-research-question-1-on-the-frequency-and-nature-of-work-with-data}

In terms of the frequency and nature of work with data, work with data
was found to be common in the summer STEM programs that made the context
for this study. A coding frame synthesized from past research carried
out in STEM domains on work with data was used to find that work with
work with data occurred from around one-quarter of the time of the
program's time (making observations) to around one-half of the program's
time (generating data and communicating findings). Data modeling was,
like making observations, less common, whereas asking questions and
generating data, like communicating findings, were relatively more
common. These findings are as expected based on past research (Lee \&
Wilkerson, in press) and given the design and goals of summer STEM
programs (Dabney et al., 2012; Elam et al., 2012), including those
participating in the present study.

In-depth qualitative showed that asking questions, generating data, and
interpreting and communicating findings, the three aspects that were
\emph{more frequent} in the programs, also were more consistent with
general aspects of learner-centered, hands-on STEM activities, and not
as consistent with work with data in particular. This suggests that
while work with data is somewhat common, qualitative analysis is an
important part of understanding youths' engagement in work with data.
More veridical forms of it are somewhat less common, occurring in around
25\% of the programs' time. While descriptive in nature, these results
present the first insight that I am aware of of the extent of work with
data in STEM enrichment programs. They suggest that, as past scholarship
(National Research Council, 2009, 2012) can provide a context for youth
to be involved in the type of scientific and engineering
practices-focused activities that can be particularly powerful for youth
(and students) in terms of their learning.

\subsection{Key findings for research question \#2 (what profiles of
engagement
emerge)}\label{key-findings-for-research-question-2-what-profiles-of-engagement-emerge}

Six profiles of engagement were identified. These were selected using a
rigorous model selection approach and through use of a sophisticated
modeling approach (LPA) and statistically software developed for this
analysis (tidyLPA). These profiles included those that were strongly
negative (\emph{Universally low}) and strongly positive (\emph{Full}),
as well as those characterized by different levels of engagement
(\emph{Only behavioral} and \emph{Only affective}) and by different
levels of the conditions of engagement (Engaged and Competent but not
Challenged). An \emph{All moderate} profile was also identified. The
profiles suggest that the experiences of youth in summer STEM programs
are variable and that the use of ESM can aid in the study of youths'
engagement. Little research has examined profiles of engagement, though
Schmidt et al. (2018) examined profiles of engagement, constructed from
items for cognitive, behavioral, and affective engagement (but not
perceptions of challenge and competence), and found six profiles, some
of which partially overlap with those found in the present study. In
particular, on the basis of the items shared between the studies, a
\emph{Universally low}, \emph{All moderate}, and \emph{Full} profile
were found in both studies. However, as these profiles are characterized
by the (uniform) level across all of the variables, this is only limited
evidence for the presence of these profiles in the larger population of
youth engaged in science and STEM-related learning activities.

\subsection{Key findings for research question \#3 (how work with data
relates to
engagement)}\label{key-findings-for-research-question-3-how-work-with-data-relates-to-engagement}

Before relations between the groups of ``predictor'' variables, work
with data and youth characteristics, and the profiles, were explored,
the amount of variability that could be explained at the program, youth,
and instructional episode levels were explored; use of cross-classified
mixed effects models were particularly helpful for this goal. The amount
of variability that could be explained at the program and instructional
episode level was small (no larger for any profile than .023, and as low
as .00 at the program level and .004 at the instructional episode level
for some profiles), while the amount of variability that could be
explained at the youth level was moderate to large (between .099 and
.427). This suggests that while there is variability in the composition
of the profiles that were identified, youth characteristics--their
pre-program thoughts, beliefs, and characteristics and their inclination
to engage in particular ways throughout the program--largely explains
the prevalence of the profiles. This also suggests that what youth do
during the programs, and the design and implementation of the programs
themselves, have little to do with how youth engage in them. This
implies that even the strongest predictor variables at these
(instructional episode and program) levels would likely not explain much
variability in the profiles (though this is not always the case, as
there are cases in which adding variables at one level can increase the
amount of variability that can be explained at another; Gelman \& Hill,
2007).

In line with what the preliminary analysis of the amount of variability
that could be explained at the youth, instructional episode, and program
levels, relations between work with data were largely not found, though
some small, statistically significant relations were identified.
Importantly, both generating and modeling data were found to be
positively related to the \emph{Full} profile, suggesting that when
youth are involved in these practices, then they are more likely to
report high levels of cognitive, behavioral, and affective engagement,
and high perceptions of competence and challenge. The effect of data
modeling was more robust than that for generating data, the latter which
should be interpreted with caution. In short, this suggests that these
activities are beneficial to youths' engagement. Both communicating and
interpreting findings and the composite measure for work with data were
positively related to the \emph{Only behavioral} profile and these
findings were fairly robust. This profile may indicate that students are
experiencing a routine engagement (and not particularly adaptive) when
they are communicating findings and being involved in work with data in
general. As there is no research on how work with data relates to
youths' engagement, the findings associated with this research question
provide some, albeit limited, evidence (and directions for future
research) for how some aspects of work with data relate to youths'
engagement.

\subsection{Key findings for research question \#4 (how youth
characteristics relate to
engagement)}\label{key-findings-for-research-question-4-how-youth-characteristics-relate-to-engagement}

Not as much in line with expectations given the preliminary analysis,
relations between youth characteristics and the profiles were found to
be small. In this way, these small relations were similar (in magnitude)
to those between work with data and the profiles. Youth with higher
pre-program interest were more likely to be \emph{Engaged and competent
but not challenged}, suggesting that youth with higher interest in STEM
are inclined to be highly engaged and good at what they are doing, but
are not challenged by the activities they experience. This could be a
function of the relationship between youths' interest and their
competence before the program, which are often strongly related
({[}add{]}); these youth, as a result of their higher interest and
competence, need more challenging activities to be more fully engaged.
This effect was fairly robust. The interaction with gender and the work
with data composite revealed a positive relationship with \emph{Full}
engagement, suggesting that the more that female youth work with data,
the more likely they are to be positively engaged. However, sensitivity
analysis revealed that this effect was not very robust, which, along
with its small magnitude, suggests that it should be interpreted with
some caution. Finding that female youth who are engaged in work with
data are more likely to be fully engaged is important, given that past
research has suggested that female students are less likely to be
engaged in STEM classes but we have limited information about what types
of instruction may best support female students to be engaged and
successful (e.g., Patall et al., 2017).

\section{Limitations of the Study and Recommendations for Future
Research}\label{limitations-of-the-study-and-recommendations-for-future-research}

This study examines youths' engagement as an outcome. Accordingly,
outcomes from engaging, such as the products of neither youths' work or
the specific cognitive capabilities they develop through their
participation, are not the focus. Thus, while some findings about how
work with data and youth characteristics were found to be associated
with different profiles of engagement, we do not have an understanding
of how engaging in more or less adaptive ways relates to these outcomes.
Examining how work with data and engagement relate to key learning,
motivational, and future goals and plans-related outcomes is a topic for
future research.

Another limitation concerns the context of the study, summer STEM
programs. While the programs that were involved in the study have many
affordances for work with data and for being highly engaging for youth,
they have some limitations, too, particularly with respect to support
work with data. Importantly, these were not programs explicitly designed
to support work with data; while such contexts are being developed, they
are not yet widespread. Learning environments that deliberately support
work with data over a long period may demonstrate different patterns of
engagement than those examined in this study because of the focus on and
sequencing of the aspects of work with data, which may make it more (or
less) cognitively, behaviorally, or affectively engaging than is
determined in this study. As Miller, Manz, Russ, Stroupe, and Berland
(2018) highlight, truly engaging STEM activities are not easily come by;
they require students to take ownership over and to make decisions about
their explorations or designs. Thus, future research may study work with
data in contexts designed to support it. A key part of this future
research may be studying both work with data and how work with data is
supported (most importantly by the instructor but also by the curriculum
and technological tools).

A related limitation is that the programs that were the focus of this
study were model programs, or those based on characteristics of
exemplary STEM enrichment programs. As a result, engagement may be
different in other STEM enrichment programs depending on characteristics
of the programs and their activities, and findings from this study
should be interpreted in terms of programs that share similar
characteristics.

A potential issue concerns the analytic approach. As noted above, the
profiles demonstrated very little variability at the program and
instructional episode level, suggesting that factors at this level would
likely not strongly predict the profiles. This could be a function of
the use of profiles and the specific variables selected. It may also be
the result of the outcome (engagement and its conditions) selected.
Other analytic approaches can be carried out to determine the viability
of the profiles approach and use of the items for engagement and its
conditions for understanding work with data.

A final limitation concerns the measures used. In particular, the
qualitative coding revealed alignment but also discrepancies between
work with data as determined from the PQA codes and the conceptual
framework for work with data. While these issues were small, they
suggest that the coding frame for work with data is a limitation of the
present study.

While these are important limitations, it is worth noting that the
modeling strategy (with the mixed effects models) in inherently a
conservative approach. Thus, while the findings detected are small, they
can be considered to be trustworthy on the basis of the way the ESM data
were analyzed. This trustworthiness is enhanced by the use of
sensitivity analysis, which showed how much of the effects could be due
to bias for them to be invalidated.

\section{Implications for Practice}\label{implications-for-practice}

\subsection{Engage youth in key aspects of work with
data}\label{engage-youth-in-key-aspects-of-work-with-data}

While limited evidence, this study suggests that generating and modeling
data in particular may be beneficial in terms of engaging youth.
Generating data in particular may be a key practice because it involves
making work with data concrete; as Lehrer and Schauble (2015) describe,
recording data in the form of ``inscriptions'' can serve as commitments
that youth make (in terms of what data were chosen to be collected and
recorded). This implication, in particular, should be interpreted with
caution, however, given the very small magnitude of the effect.
Similarly, data modeling has been described as \emph{the} central
scientific and engineering practice (Schwarz et al., 2009; Lehrer \&
Schauble, 2015; Weisberg, 2012), and its relations with full engagement
provides some actionable evidence for its importance in the context of
summer STEM programs.

Practically, youth activity leaders (in summer STEM and other STEM
enrichment contexts) and teachers (in formal learning environments) can
best include the beneficial practices of generating and modeling data
not in isolation, but rather through involving youth and learners in
complete cycles of investigation. This aligns with both foundational and
contemporary research on work with data in education (Berland et al.,
2018; McNeill \& Berland, 2017; Hancock et al., 1992; Lee \& Wilkerson,
2018). Recent curricular reform efforts also suggest that the best way
to engage learners in particular practices is through the process of
identifying a question or problem, marshaling sources of data that can
be used to figure out what is happening, and developing model-based
explanations that are then communicated (or even used in an argument;
National Governors Association, 2013; National Research Council, 2012;
NGSS Lead States, 2013). With respect to work with data in particular,
youth activity leaders and teachers can use the findings from this study
as a starting point to consider how engaging in work with data may also
prepare learners to think of, understand, and take action based on data
in their day-to-day lives. Many questions or problems learners face may
involve data that can be meaningfully incorporated into engaging
learning activities.

\subsection{Leverage the affordances of summer STEM and other STEM
enrichment
programs}\label{leverage-the-affordances-of-summer-stem-and-other-stem-enrichment-programs}

Another implication for practice concerns the affordances (and
constraints) of summer STEM and other STEM enrichment programs. One
affordance of these programs relevant to these informal and to K-12
learning environments concerns selecting activities that are engaging to
youth. For example, in the \emph{Marine Investigators}, youth
participated in activities designed to help them understand water
quality in their ecosystem. Youth collected trash from sites around
their community (in different ``districts'') and then brought the trash
and recyclable plastic back to the classroom. Then, the youth activity
leaders asked students to figure out how much plastic enters local
waterways. As a part of this activity, youth activity leaders asked
students not only to determine the quantity of trash that entered the
waterways, but asked students about \emph{why} they used math in
particular ways (i.e., adding the quantity of trash collected and then
extrapolating from this quantity to the amount from across the entire
city over the course of the year). This appeared to be a powerful
activity, one that was coded as involving all five aspects of work with
data according to the measures for instructional support for work with
data; this type of activity seemed to suggest that instructional support
for work with data may impact youth's engagement.

Another affordance concerned the relevance of the program to youth's
lives. For example, in the \emph{Building Mania} program, youth are
involved in engineering design (i.e., identifying a problem and
designing a solution), particularly around the use of simple machines.
In a day in the classroom setting, youth are creating, testing, and
revising catapults. In the next day, youth visit an area University, and
are led in a discussion by a physicist who works with particle
colliders. In this example, the expertise of the physicist, who
explicitly mentions the benefits of engaging in the engineering design
process and the importance of combining engineering to addressing
problems (such as mitigating the damage of earthquakes), seems to be
highly relevant to what youth are doing in their class. In these two
days of class, youth are engaged in different aspects of work with data
as indicated by the codes for instructional support for work with data
(collecting data on the efficacy of their designs in the classroom day,
and asking questions in the subsequent day, particularly); these seem to
suggest, like the example of work work with data from the \emph{Marine
Investigators} program, affordances of work with data for summer STEM
programs.

\subsection{Consider the constraints of summer STEM and other STEM
enrichment
programs}\label{consider-the-constraints-of-summer-stem-and-other-stem-enrichment-programs}

There are also constraints to summer STEM and other STEM enrichment
programs. For example, youth activity leaders faced challenges linking
activities as part of a complete cycle of investigation. For example, in
the \emph{Ecosphere} program, youth collected water samples in the
field. They then brought these samples to the classroom and tested the
water, involving students in both collecting and, to a degree,
generating data (by noting the pH levels of the water). However, later
in the day, youth created a small-scale model (with inclined trays of
dirt, rocks, and plants) of an ecosystem, in which they added food
coloring to determine the impacts of chemicals and acid rain. Youth then
interpreted and discussed these findings, but did not connect the
discussion to the water samples youth collected and tested earlier. This
activity presented an opportunity for deeper engagement, in which youth
could interpret and communicate findings related to the state of the
water in their ecosystem, but, instead, it was potentially limiting in
terms of youth's engagement in work with data.

Another constraint related to the challenge of linking activities
concerned what the programs focused on. For example, the
mathematics-focused programs, such as the \emph{Adventures in
Mathematics} program, the youth activity leaders recognizing that youth
had difficulty solving equations, used duct tape and a ``hippity
hoppity'', building on an earlier activity in which youth considered
what constituted a rate, on how many ``hops'' it would take someone to
move from one end of the line of duct tape to the other; the youth
activity leader than asked youth to consider how far they could move in
one hop and to consider how they could find out many hops it would take,
using a mathematical equation. In this activity, youth were supported to
approach mathematics problem-solving in creative ways. However, apart
from data modeling, other aspects of work with data were rarely present,
and most of the data that youth worked with was provided by the teacher
or considered in the abstract. Programs focused on science or
engineering, similarly, emphasized other aspects of work with data: The
science-focused programs (\emph{Island Explorers}, \emph{The Ecosphere},
and \emph{Marine Investigators}) all emphasized collecting and
generating data, but data, particularly the data collected or generated,
was rarely modeled or interpreted. In the engineering-focused programs
(\emph{Uptown Architecture}, \emph{Crazy Machines}, and \emph{Dorchester
House}, youth often collected data that resulted from their engineering
designs, and communicated and interpreted their findings, but, did not
generate data, and, accordingly, (and like the science-focused programs)
did not model data as a regular part of their activities. This finding
suggests that while work with data may have been common overall,
different aspects of instructional support for work with data were
emphasized to different degrees based on the focus of the program.

\section{Conclusion}\label{conclusion}

Each of the disciplines that contribute to STEM learning involve work
with data and how youth and students work with data in engaging ways is
a concern of researchers and practitioners. While past research has
focused on what aspects of work with data learners are involved in with
respect to work with data, or specific conceptual outcomes from working
with data, little research has considered youths' experience of working
with data. In this study, engagement was used as a lens to understand
the experience of youth working with data in the context of nine summer
STEM programs. In particular, five aspects of work with data, a) asking
questions, b) observing phenomena, c) constructing measures and
generating data, d) data modeling, and e) interpreting and communicating
findings, were identified from video-recordings of the programs. These
codes were then used to predict profiles, or distinct groups on the
basis of different levels, of youths' cognitive, behavioral, and
affective engagement, and two other variables, youths' perceptions of
challenge and competence. These measures were obtained using an
innovative method, ESM, that provides some access to youths' experience
in-the-moment of the activities they were involved in during the
program.

Findings indicate that work with data occurs regularly in the programs
and that there are some examples of ambitious activities centered on
working with real-world data (and examples in which the work with data
is not fully aligned with youth-driven work with data). Six profiles of
engagement were identified, representing different configurations of the
three dimensions of engagement and its conditions. Relations of work
with data and youth characteristics (pre-program interest in STEM and
youths' gender and status in terms of being a member of
under-represented groups in STEM) were, overall, not strongly related
with the profiles of engagement, though some key findings were
identified. Generating and modeling data were both related to the most
potentially beneficial profile, one characterized by high levels of all
five of the variables used to create the profiles. Female youth who were
involved in work with data (at the instructional episode level) to a
greater extent were also more likely to be fully engaged. This study
suggests that work with data has purchase as an organizing set of
activities for STEM can have some benefits in terms of understanding the
nature of what youth do in summer STEM programs. In addition, this study
shows that ESM and engagement can be used to understand youths'
experiences. Data--and who is able to work with data--have important
roles in STEM learning and in society; efforts to understand and support
learners engaging in these ambitious activities should be encouraged and
expanded.

\chapter{References}\label{references}

\setlength{\parindent}{-0.2in} \setlength{\leftskip}{0.2in}
\setlength{\parskip}{8pt} \noindent

Akiva, T. (2005). Turning training into results: The new youth program
quality assessment. High/Scope Resource, 24(2), 21-24.

Bergman, L. R., \& Magnusson, D. (1997). A person-oriented approach in
research on developmental psychopathology. Development and
psychopathology, 9(2), 291-319.

Bergman, L. R., Magnusson, D., \& El Khouri, B. M. (2003). Studying
individual development in an interindividual context: A person-oriented
approach. Psychology Press.

Berland, L. K., Schwarz, C. V., Krist, C., Kenyon, L., Lo, A. S., \&
Reiser, B. J. (2016). Epistemologies in practice: Making scientific
practices meaningful for students. Journal of Research in Science
Teaching, 53(7), 1082-1112.

Bielik, T., \& Yarden, A. (2016). Promoting the asking of research
questions in a high-school biotechnology inquiry-oriented program.
International Journal of STEM Education, 3(1), 15.

Breckenridge, J. N. (2000). Validating cluster analysis: Consistent
replication and symmetry. Multivariate Behavioral Research, 35(2),
261-285.

Bystydzienski, J. M., Eisenhart, M., \& Bruning, M. (2015). High school
is not too late: Developing girls' interest and engagement in
engineering careers. Career Development Quarterly, 63(1), 88--95.
\url{http://doi.org/10.1002/j.2161-0045.2015.00097.x}

Cohen, J. (1992). A power primer. Psychological Bulletin, 112(1), 155.

National Governors Association Center for Best Practices, Council of
Chief State School Officers. (2010). Common Core State Standards for
Mathematics. Washington, DC: National Governors Association Center for
Best Practices and the Council of Chief State School Officers.

Corpus, J. H., \& Wormington, S. V. (2014). Profiles of intrinsic and
extrinsic motivations in elementary school: A longitudinal analysis. The
Journal of Experimental Education, 82(4), 480-501.

Csikszentmihalyi, M. (1990). Flow: The psychology of optimal
performance. Cambridge, England: Cambridge University Press.

Csikszentmihalyi, M. (1997). Finding flow: The psychology of engagement
with everyday life. New York, NY: Basic Books.

Creswell, J. W., Plano Clark, V. L., Gutmann, M. L., \& Hanson, W. E.
(2003). Advanced mixed methods research designs. In A. Tashakkori \& C.
Teddlie (Eds.), Handbook of mixed methods in social and behavioral
research (pp.~209--240). Thousand Oaks, CA: Sage.

English, L. D. (2012). Data modelling with first-grade students.
Educational Studies in Mathematics, 81(1), 15-30.

Finzer, W. (2013). The data science education dilemma. Technology
Innovations in Statistics Education, 7(2), p.~1-9.\\
Forum for Youth Investment. (2012). Youth Program Quality Assessment.
Washington, DC: The Forum for Youth Investment Franklin, C., Kader, G.,
Mewborn, D., Moreno, J., Peck, R., Perry, M., \& Scheaffer, R. (2007).
Guidelines for assessment and instruction in statistics education
(GAISE) report. Alexandria, VA: American Statistical Association.

Fredricks, J. A., \& McColskey, W. (2012). The measurement of student
engagement: A comparative analysis of various methods and student
self-report instruments. In S. L. Christenson, A. L. Reschly, \& C.
Wylie (Eds.), The handbook of research on student engagement
(pp.~763--782). New York: Springer Science.
\url{https://doi.org/10.1007/978-1-4614-2018-7_37}

Fredricks, J. A., Blumenfeld, P. C., \& Paris, A. H. (2004). School
engagement: Potential of the concept, state of the evidence. Review of
Educational Research, 74(1), 59-109.

Fredricks, J. A., Filsecker, M., \& Lawson, M. A. (2016). Student
engagement, context, and adjustment: Addressing definitional,
measurement, and methodological issues. Learning \& Instruction, 43,
1-4.

Gelman, S. A., \& Markman, E. M. (1987). Young children's inductions
from natural kinds: The role of categories and appearances. Child
Development, 58(6), 1532-1541.

Gopnik, A., \& Sobel, D. M. (2000). Detecting blickets: How young
children use information about novel causal powers in categorization and
induction. Child Development, 71(5), 1205-1222.

Gopnik, A., Sobel, D. M., Schulz, L. E., \& Glymour, C. (2001). Causal
learning mechanisms in very young children: two-, three-, and
four-year-olds infer causal relations from patterns of variation and
covariation. Developmental Psychology, 37(5), 620.

Greene, B. A. (2015). Measuring cognitive engagement with self-report
scales: Reflections from over 20 years of research. Educational
Psychologist, 50(1), 14-30.

Greene, K. M., Lee, B., Constance, N., \& Hynes, K. (2013). Examining
youth and program predictors of engagement in out-of-school time
programs. Journal of Youth and Adolescence, 42(10), 1557-1572.

Hancock, C., Kaput, J. J., \& Goldsmith, L. T. (1992). Authentic inquiry
with data: Critical barriers to classroom implementation. Educational
Psychologist, 27(3), 337-364.

Harring, J. R., \& Hodis, F. A. (2016). Mixture modeling: Applications
in educational psychology. Educational Psychologist, 51(3-4), 354-367.

Hasson, E., \& Yarden, A. (2012). Separating the research question from
the laboratory techniques: Advancing high‐school biology teachers'
ability to ask research questions. Journal of Research in Science
Teaching, 49(10), 1296-1320.

Hayenga, A. O., \& Corpus, J. H. (2010). Profiles of intrinsic and
extrinsic motivations: A person-centered approach to motivation and
achievement in middle school. Motivation and Emotion, 34(4), 371-383.

Hektner, J. M., Schmidt, J. A., \& Csikszentmihalyi, M. (2007).
Experience sampling method: Measuring the quality of everyday life.
Sage.

Jahnukainen, M. (2010). Extreme cases. Encyclopedia of Case Study
Research. Thousand Oaks, CA: Sage.

Konold, C., \& Pollatsek, A. (2002). Data analysis as the search for
signals in noisy processes. Journal for Research in Mathematics
Education, 33(4), 259-289.

Lauer, P. A., Akiba, M., Wilkerson, S. B., Apthorp, H. S., Snow, D., \&
Martin-Glenn, M. L. (2006). Out-of-school-time programs: A meta-analysis
of effects for at-risk students. Review of educational research, 76(2),
275-313.

Lee, H. S., Angotti, R. L., \& Tarr, J. E. (2010). Making comparisons
between observed data and expected outcomes: students' informal
hypothesis testing with probability simulation tools. Statistics
Education Research Journal, 9(1), 68-96.

Lee, H., \& Hollebrands, K. (2008). Preparing to teach mathematics with
technology: An integrated approach to developing technological
pedagogical content knowledge. Contemporary Issues in Technology and
Teacher Education, 8(4), 326-341.

Lehrer, R., \& Romberg, T. (1996). Exploring children's data modeling.
Cognition and Instruction, 14(1), 69-108.

Lehrer, R., \& Schauble, L. (2004). Modeling natural variation through
distribution. American Educational Research Journal, 41(3), 635-679.

Lehrer, R. \& Schauble, L. (2015). Developing scientific thinking. In L.
S. Liben \& U. Müller (Eds.), Cognitive processes. Handbook of child
psychology and developmental science (Vol. 2, 7th ed., pp.~671-174).
Hoboken, NJ: Wiley.

Lehrer, R., Kim, M. J., \& Jones, R. S. (2011). Developing conceptions
of statistics by designing measures of distribution. ZDM, 43(5),
723-736.

Lehrer, R., Kim, M. J., \& Schauble, L. (2007). Supporting the
development of conceptions of statistics by engaging students in
measuring and modeling variability. International Journal of Computers
for Mathematical Learning, 12(3), 195-216.

Lesh, R., Middleton, J. A., Caylor, E., \& Gupta, S. (2008). A science
need: Designing tasks to engage students in modeling complex data.
Educational Studies in Mathematics, 68(2), 113-130.

Linnansaari, J., Viljaranta, J., Lavonen, J., Schneider, B., \&
Salmela-Aro, K. (2015). Finnish Students Engagement in Science Lessons.
NorDiNa: Nordic Studies in Science Education, 11(2), 192-206. Retrieved
from
\url{https://www.journals.uio.no/index.php/nordina/article/view/2047}

Lovett, M. C., \& Shah, P. (2007). Preface. In M. C. Lovett \& P. Shah
(Eds.), Thinking with data (pp.~x-xx {[}requested book through ILL to
confirm page \#s{]}). New York, NY: Lawrence Erlbaum.

Magnusson, D., \& Cairns, R. B. (1996). Developmental science: Toward a
unified framework. Cambridge, England: Cambridge University Press.

McNeill, K. L., \& Berland, L. (2017). What is (or should be) scientific
evidence use in k‐12 classrooms? Journal of Research in Science
Teaching, 54(5), 672-689.

Muthén, B. (2004). Latent variable analysis. The Sage handbook of
quantitative methodology for the social sciences. Thousand Oaks, CA:
Sage Publications, 345-68.

Muthén, L. K., \& Muthén, B. O. (1997-2017). Mplus User's Guide. Los
Angeles, CA: Muthén \& Muthén.

NGSS Lead States. (2013). Next generation science standards: For states,
by states. Washington, DC: National Academies Press.

Nolen, S. B., Horn, I. S., \& Ward, C. J. (2015). Situating motivation.
Educational Psychologist, 50(3), 234-247. Patall, E. A., Vasquez, A. C.,
Steingut, R. R., Trimble, S. S., \& Pituch, K. A. (2016). Daily
interest, engagement, and autonomy support in the high school science
classroom. Contemporary Educational Psychology, 46, 180-194.

Patall, E. A., Steingut, R. R., Vasquez, A. C., Trimble, S. S., Pituch,
K. A., \& Freeman, J. L. (2017). Daily Autonomy Supporting or Thwarting
and Students' Motivation and Engagement in the High School Science
Classroom. Journal of Educational Psychology. Advance online
publication. \url{http://dx.doi.org/10.1037/edu0000214}

Pekrun, R., \& Linnenbrink-Garcia, L. (2012). Academic emotions and
student engagement. In S. L. Christenson, A. L. Reschly, \& C. Wylie
(Eds.), Handbook of research on student engagement (pp.~259-292). New
York, NY: Springer.

Petrosino, A., Lehrer, R., \& Schauble, L. (2003). Structuring error and
experimental variation as distribution in the fourth grade. Mathematical
Thinking and Learning, 5 (2\&3), 131-156.

Piaget, J., \& Inhelder, B. (1969). The psychology of the child. New
York, NY: Basic Books.\\
Pöysä, S., Vasalampi, K., Muotka, J., Lerkkanen, M. K., Poikkeus, A. M.,
\& Nurmi, J. E. (2017). Variation in situation-specific engagement among
lower secondary school students. Learning and Instruction.
\url{http://dx.doi.org/10.1016/j.learninstruc.2017.07.007}

Rosenberg, J. M. (2018). Comparing mplus and mclust output. Retrieved
from
\url{https://jrosen48.github.io/r-markdown/comparing-mplus-mclust.html}

Salmela-Aro, K., Moeller, J., Schneider, B., Spicer, J., \& Lavonen, J.
(2016). Integrating the light and dark sides of student engagement using
person-oriented and situation-specific approaches. Learning and
Instruction, 43, 61-70.

Salmela-Aro, K., Muotka, J., Alho, K., Hakkarainen, K., \& Lonka, K.
(2016). School burnout and engagement profiles among digital natives in
Finland: A person-oriented approach. European Journal of Developmental
Psychology, 13(6), 704-718.

Schneider, B., Krajcik, J., Lavonen, J., Salmela‐Aro, K., Broda, M.,
Spicer, J., \ldots{} \& Viljaranta, J. (2016). Investigating optimal
learning moments in US and Finnish science classes. Journal of Research
in Science Teaching, 53(3), 400-421.

Schmidt, J. A., Rosenberg, J. M., Beymer, P. (advance online
publication). A person-in-context approach to student engagement in
science: Examining learning activities and choice. Journal of Research
in Science Teaching. \url{https://dx.doi.org/10.1002/tea.21409}

Schwarz, N., Kahneman, D., \& Xu, J. (2009). Global and episodic reports
of hedonic experience. In R. Belli, D. Alwen, \& F. Stafford (Eds.),
Using calendar and diary methods in life events research (pp.~157-174).
Newbury Park, CA: Sage.

Sfard, A. (1998). On two metaphors for learning and the dangers of
choosing just one. Educational Researcher, 27(2), 4-13.

Shernoff, D. J., Csikszentmihalyi, M., Schneider, B., \& Shernoff, E. S.
(2003). Student engagement in high school classrooms from the
perspective of flow theory. School Psychology Quarterly, 18(2), 158-176.

Shernoff, D. J., Kelly, S., Tonks, S. M., Anderson, B., Cavanagh, R. F.,
Sinha, S., \& Abdi, B. (2016). Student engagement as a function of
environmental complexity in high school classrooms. Learning and
Instruction, 43, 52-60.\\
Shumow, L., \& Schmidt, J. A. (2013). STEM interest and engagement (STEM
I.E.). National Science Foundation proposal for award number 1421198.

Sinatra, G. M., Heddy, B. C., \& Lombardi, D. (2015). The challenges of
defining and measuring student engagement in science. Educational
Psychologist, 50(1), 1-13. \url{doi:10.1080/00461520.2014.1002924}

Singh, K., Granville, M., \& Dika, S. (2002). Mathematics and science
achievement: Effects of motivation, interest, and academic engagement.
The Journal of Educational Research, 95(6), 323-332.

Shernoff, D. J., \& Schmidt, J. A. (2008). Further Evidence of an
Engagement--Achievement Paradox Among U.S. High School Students. Journal
of Youth and Adolescence, 37(5), 564--580.
\url{http://doi.org/10.1007/s10964-007-9241-z}

Shumow, L., Schmidt, J. A., \& Zaleski, D. J. (2013). Multiple
perspectives on student learning, engagement, and motivation in high
school biology labs. The High School Journal, 96(3), 232-252.

Skinner, E. A., \& Pitzer, J. (2012). Developmental dynamics of
engagement, coping, and everyday resilience. In S. Christenson, A.
Reschly, \& C. Wylie (Eds.), Handbook of Research on Student Engagement
(pp.~21-45). New York: Springer Science.

Skinner, E. A., Kindermann, T. A., \& Furrer, C. J. (2009). A
motivational perspective on engagement and disaffection:
Conceptualization and assessment of children's behavioral and emotional
participation in academic activities in the classroom. Educational and
Psychological Measurement, 69(3), 493-525.

Skinner, E., Furrer, C., Marchand, G., \& Kindermann, T. (2008).
Engagement and disaffection in the classroom: Part of a larger
motivational dynamic? Journal of Educational Psychology, 100(4), 765.

Smith, C., Akiva, T., Sugar, S., Lo, Y. J., Frank, K. A., Peck, S. C.,
Cortina, K. S., \& Devaney, T. (2012).Continuous quality improvement in
afterschool settings: Impact findings from the Youth Program Quality
Intervention study. Washington, DC: The Forum for Youth Investment.

Steinley, D., \& Brusco, M. J. (2011). Evaluating mixture modeling for
clustering: recommendations and cautions. Psychological Methods, 16(1),
63.

Stohl, H., \& Tarr, J. E. (2002). Developing notions of inference using
probability simulation tools. The Journal of Mathematical Behavior,
21(3), 319-337.

Stroupe, D. (2014). Examining classroom science practice communities:
How teachers and students negotiate epistemic agency and learn
science‐as‐practice. Science Education, 98(3), 487-516.

Strati, A. D., Schmidt, J. A., \& Maier, K. S. (2017). Perceived
challenge, teacher support, and teacher obstruction as predictors of
student engagement. Journal of Educational Psychology, 109(1), 131-147.

Trevors, G. J., Kendeou, P., Bråten, I., \& Braasch, J. L. (2017).
Adolescents' epistemic profiles in the service of knowledge revision.
Contemporary Educational Psychology, 49, 107-120.

Turner, J. C., \& Meyer, D. K. (2000). Studying and understanding the
instructional contexts of classrooms: Using our past to forge our
future. Educational Psychologist, 35(2), 69-85.

van Rooij, E. C., Jansen, E. P., \& van de Grift, W. J. (2017).
Secondary school students' engagement profiles and their relationship
with academic adjustment and achievement in university. Learning and
Individual Differences, 54, 9-19.

Vandell, D. L., Hall, V., O'Cadiz, P., \& Karsh, A. (2012). Piloting
outcome measures for summer learning initiative programs. Final report
to the David and Lucile Packard Foundation, Children, Families, and
Communities Program. Retrieved from
\url{http://faculty.sites.uci.edu/childcare/files/2013/07/SL-Outcomes-2011-Pilot_Edited_8.19.pdf}

Wang, M. T., \& Eccles, J. S. (2012). Social support matters:
Longitudinal effects of social support on three dimensions of school
engagement from middle to high school. Child Development, 83(3),
877-895.

Wang, M. T., \& Holcombe, R. (2010). Adolescents' perceptions of school
environment, engagement, and academic achievement in middle school.
American Educational Research Journal, 47(3), 633-662.

Westfall, J., Kenny, D. A., \& Judd, C. M. (2014). Statistical power and
optimal design in experiments in which samples of participants respond
to samples of stimuli. Journal of Experimental Psychology: General,
143(5), 2020-2045.

Wild, C. J., \& Pfannkuch, M. (1999). Statistical thinking in empirical
enquiry. International Statistical Review, 67(3), 223-248.

Wilkerson, M. H., Andrews, C., Shaban, Y., Laina, V., \& Gravel, B. E.
(2016). What's the technology for? Teacher attention and pedagogical
goals in a modeling-focused professional development workshop. Journal
of Science Teacher Education, 27(1), 11-33.

Wilkerson, M. H. \& Fenwick, M. (2017). The practice of using
mathematics and computational thinking. In C. V. Schwarz, C. Passmore,
\& B. J. Reiser (Eds.), Helping Students Make Sense of the World Using
Next Generation Science and Engineering Practices. Arlington, VA:
National Science Teachers' Association Press. pp.~181-204.

Witherington, D. C. (2015). Dynamic systems in developmental science. In
W. F. Overton \& P. C. M. Molenaar (Vol. Eds.) \& R. M. Lerner (Ed.),
Handbook of child psychology and developmental science. Vol. 1: Theory
\& method (7th ed., pp.~63-112). Hoboken, NJ: Wiley.

Wormington, S. V., \& Linnenbrink-Garcia, L. (advance online
publication). A new look at multiple goal pursuit: The promise of a
person-centered approach. Educational Psychology Review.
\url{doi:10.1007/s10648-016-9358-2}

\chapter{Appendix}\label{appendix}

\section{Appendix A: STEM-PQA
alignment}\label{appendix-a-stem-pqa-alignment}

\begin{landscape}\begin{table}

\caption{\label{tab:unnamed-chunk-14}Alignment of codes for instructional support for work with data and the STEM-PQA}
\centering
\begin{tabular}[t]{lll}
\toprule
Work.With.Data & Description & STEM.PQA\\
\midrule
Asking questions or defining problems & Discussing and exploring topics to investigate and pose questions. & Predict, conjecture, or hypothesize\\
Making observations & Watching and noticing what is happening with respect to the phenomena or problem being investigated. & Classify or abstract\\
Generating data & Figuring out how or why to inscribe an observation as data and generating coding frames or measurement tools. & Collect data or measure; Highlight precision and accuracy\\
 &  & \\
Data modeling & Understanding and explaining phenomena using models of the data that account for variability or uncertainty. & Simulate, experiment, or model\\
Interpreting and communicating findings & Discussing and sharing and presenting findings. & Analyze; Use symbols or models\\
 &  & \\
 &  & \\
\bottomrule
\end{tabular}
\end{table}
\end{landscape}

\subsection{Appendix B: Program
descriptions}\label{appendix-b-program-descriptions}

\begin{landscape}\begin{table}

\caption{\label{tab:unnamed-chunk-15}Program (with pseudonyms) descriptions}
\centering
\begin{tabular}[t]{ll}
\toprule
Program.Name & Program.Description\\
\midrule
Island Explorers & A science-focused program that aims to help youth develop expertise on one species found in the local ecosystem by reading and writing about related content for up to an hour per day; undertaking data collection and analysis tasks to learn about the local ecosystem and how to communicate scientific data; developing vocabulary about the local ecosystem; using art to learn and communicate information; and publishing a book illustrating important elements of the species being studied. Located in both the classroom and local ecosystem. 27 students who are rising 6th graders. Youth spend the morning in more academically-oriented sessions in a classroom setting, while afternoon sessions involved STEM-oriented enrichment sessions taking place outside (the program was associated with Outward Bound) with an emphasis on exploration of the local ecosystem.\\
The Ecosphere & A science-focused program that aims to help youth to explore the marine life of Narragansett Bay. Efforts were undertaken to build youth content knowledge in the areas of ecosystem preservation, marine biology, and water quality, and related skills, such as questioning, showing initiative, data collection, measuring, maintaining an ecosystem, and analyzing water samples. Located in a classroom setting, shoreline, and science education center. 27 youth who are rising 6th to 9th graders. Youth attended programming in a classroom at an area middle school and in a field-based setting on alternating days. Field-based settings included a science education center at a community-based organization and field trips to sites in the community related to the program's focus.\\
 & \\
Zoology Partners & A science-focused program that aims to support youth's development of content knowledge related to the issue of endangered species, including how species become endangered, processes for monitoring ecosystem viability and population levels, solutions to prevent species from becoming endangered, and approaches to reviving populations that are currently endangered. Located in the classroom as well as zoos, parks, and other natural areas. 25 youth who are rising 6th to 9th graders. Youth attended programming in a classroom at an area middle school and in a field-based setting on alternating days. Field-based settings included a local zoo and field trips to sites in the community related to the program's focus.\\
Marine Investigators & A science-focused program that aims to provide youth with opportunities to learn about and experience Narragansett Bay; examine human impacts on the local ecosystem, including how the geography of the Bay helped influence human history and how the history of humans along the shoreline has impacted the Bay, and begin the process of cultivating a sense of stewardship among participating youth for caring for and protecting the Bay in the future. Located in the classroom, shoreline along the bay, ship on the bay, and various field locations associated with bay health. 19 youth who are rising 7th to 9th graders. Youth attended programming in a classroom at an area middle school and in a field-based setting on alternating days. Field-based settings included the local bay shoreline, a voyage on a marine education ship researching in the Bay, and field trips to sites in the community related to the program's focus. During the span of the program, youth had the opportunity to participate in both a water quality research study.\\
Comunidad de Aprendizaje & A STEM-focused program that aims to help youth improve basic skills in mathematics and develop an interest in STEM content and entrepreneurship. Primarily in the classroom setting. 33 students who are rising 5th to 8th graders. Morning sessions are characterized by direct instruction in mathematics for individual grade levels and mixed grade level afternoon enrichment sessions in either robotics or dance. The direct instruction component of the programs was organized around a theme of promoting entrepreneurship with the goal of helping participating youth better see the relevance of mathematics to future career goals and opportunities.\\
Jefferson House & A STEM-focused program that aims to support youth's development of basic math skills, the program was primarily focused on helping youth develop problem solving, self-improvement, and critical thinking skills. Located in a classroom. 11 youth who are rising 7th graders. The youth spent the morning in more academically-oriented sessions in a classroom setting focusing on basic skill development, while afternoon sessions involved STEM-oriented enrichment sessions involving media, art, and nutrition. Enrichment offerings varied by day, with math sessions occurring twice per week, alternating with academically oriented sessions in the am that were oriented at supporting skill development in English/language arts.\\
 & \\
Uptown Architecture & An engineering-focused program that aims to support youth's participation in a process to design and build an outdoor learning space for use at the middle school where the program was housed. A key focus of the program was to provide youth with the opportunity to use design thinking as a problem-solving tool and have the experience of affecting their community positively through the design/build process. Located in a classroom, building shop, and various field locations. 18 youth who were rising 6th to 9th graders. Youth attended programming in a classroom at an area middle school and in a building shop located at a community-based organization on alternating days, while also taking field trips to locations associated with the program's overall theme.\\
Building Mania & An engineering-focused program that aims to provide youth with the opportunity to experiment with designing and using simple machines. A goal of the program is to have youth engage in the engineering design process by determining a need, brainstorming possible designs, selecting a design, planning and drawing out the design, creating and testing and revising it, and producing a final machine. Located in the classroom, design labs, and other local locations. 24 youth who are rising 6th to 9th graders. Youth attended programming in a classroom at an area middle school and a field-based setting on alternating days. Field-based settings included a design lab at a community-based organization and field trips to sites in the community related to the program's focus.\\
Adventures in Mathematics & A mathematics-focused program that aims to help youth to develop the basic math skills and prevent summer learning loss among participating youth through direct instruction and participation in math-related games. Located primarily in the classroom. 20 youth who are rising 8th to 10th graders. Youth participated in direct instructions in mathematics and math-related games in small groups. Program content was aligned with the state's standards in mathematics.\\
\bottomrule
\end{tabular}
\end{table}
\end{landscape}

\subsection{Appendix C: Work with data by
program}\label{appendix-c-work-with-data-by-program}

\begin{table}

\caption{\label{tab:unnamed-chunk-16}Proportion of signals for which each of the aspects of work with data was present}
\centering
\begin{tabular}[t]{lrr}
\toprule
Aspect of Work With Data & Proportion & N\\
\midrule
Asking Questions & 0.389 & 92\\
Making Observations & 0.258 & 61\\
Generating Data & 0.453 & 107\\
Data Modeling & 0.288 & 68\\
Communicating Findings & 0.470 & 111\\
\bottomrule
\end{tabular}
\end{table}

\begin{landscape}\begin{table}

\caption{\label{tab:unnamed-chunk-16}Proportion of signals for which each of the aspects of work with data was present by program}
\centering
\begin{tabular}[t]{lrrrrrr}
\toprule
Variable & Asking & Observing & Generating & Modeling & Communicating & Total Segments\\
\midrule
Island Explorers & 0.312 & 0.375 & 0.438 & 0.250 & 0.375 & 16\\
The Ecosphere & 0.625 & 0.417 & 0.500 & 0.292 & 0.500 & 24\\
Zoology Partners & 0.250 & 0.167 & 0.125 & 0.167 & 0.208 & 24\\
Marine Investigators & 0.458 & 0.333 & 0.250 & 0.375 & 0.542 & 24\\
Comunidad de Aprendizaje & 0.327 & 0.182 & 0.400 & 0.273 & 0.327 & 55\\
Jefferson House & 0.167 & 0.083 & 0.542 & 0.458 & 0.750 & 24\\
Uptown Architecture & 0.375 & 0.208 & 0.708 & 0.167 & 0.292 & 24\\
Building Mania & 0.333 & 0.208 & 0.375 & 0.333 & 0.500 & 24\\
Adventures in Mathematics & 0.583 & 0.292 & 0.542 & 0.458 & 0.750 & 24\\
\bottomrule
\end{tabular}
\end{table}
\end{landscape}

\subsection{Appendix D: Model specifications
details}\label{appendix-d-model-specifications-details}

Here, the six models that are possible to specify in LPA are described
in terms of how the variables used to create the profiles are estimated.
Note that \emph{p} represents different profiles and each
parameterization is represented by a 4 x 4 covariance matrix and
therefore would represent the parameterization for a four-profile
solution. In all of the models, the means are estimated freely in the
different profiles. Imagine that each row and column represents a
different variable, i.e., the first row (and column) represents broad
interest, the second enjoyment, the third self-efficacy, and the fourth
another variable, i.e., future goals and plans. Models 1 and 3 meet the
assumption of independence, that is, that, after accounting for their
relations with the profile, the variables used to estimate the profiles
are independent (Collins \& Lanza, 2010). They estimate variable
variances but do not estimate covariances (i.e., as can be seen, the
covariance matrices are ``diagonal,'' without any off-diagonal
parameters that are estimated). These models are estimated by default in
MPlus, although these assumptions can be relaxed (Muthen \& Muthen,
2017). Importantly, this does not mean the variables used to create the
profile are assumed to be not related; as Collins and Lanza (2010)
explain:

\begin{quote}
The local independence assumption refers only to conditioning on the
latent variable. It does not imply that in a data set that is to be
analyzed, the observed variables are independent. In fact, it is the
relations among the observed variables that are explained by the latent
classes. An observed data set is a mixture of all the latent classes.
Independence is assumed to hold only within each latent class, which is
why it is called ``local''.
\end{quote}

Despite the assumption of independence, as Collins and Lanza (2010),
Muthen and Muthen (2017), and others (i.e., Pastor et al., 2007; Vermunt
\& Magidson, 2002) note, it can be lifted to improve model fit, though
these models without the assumption of independence may be better
described as general or Gaussian mixture models (Fraley et al., 2017).

\subsubsection{Varying means, equal variances, and covariances fixed to
0 (model
1)}\label{varying-means-equal-variances-and-covariances-fixed-to-0-model-1}

In this model, which corresponds to the mclust model wit the name
``EEI'', the variances are estimated to be equal across profiles,
indicated by the absence of a p subscript for any of the diagonal
elements of the matrix. The covariances are constrained to be zero, as
indicated by the 0's between every combination of the variables. Thus,
this model is highly constrained but also parsimonious: the profiles are
estimated in such a way that the variables' variances are identical for
each of the profiles, and the relationships between the variables are
not estimated. In this way, less degrees of freedom are taken used to
explain the observations that make up the data. However, estimating more
parameters--as in the other models--may better explain the data,
justifying the addition in complexity that their addition involves (and
their reduction in degrees of freedom).

\[
\left[ \begin{matrix} { \sigma  }_{ 1 }^{ 2 } & 0 & 0 & 0 \\ 0 & { \sigma  }_{ 2 }^{ 2 } & 0 & 0 \\ 0 & 0 & { \sigma  }_{ 3 }^{ 2 } & 0 \\ 0 & 0 & 0 & { \sigma  }_{ 4 }^{ 2 } \end{matrix} \right] 
\]

\subsubsection{Varying means, equal variances, and equal covariances
(model
2)}\label{varying-means-equal-variances-and-equal-covariances-model-2}

This model corresponds to the mclust model ``EEE''. In this model, the
variances are still constrained to be the same across the profiles,
although now the covariances are estimated (but like the variances, are
constrained to be the same across profiles). Thus, this model is the
first to estimate the covariance (or correlations) of the variables used
to create the profiles, thus adding more information that can be used to
better understand the characteristics of the profiles (and, potentially,
better explain the data).

\[
\left[ \begin{matrix} { \sigma  }_{ 1 }^{ 2 } & { \sigma  }_{ 21 } & { \sigma  }_{ 31 } & { \sigma  }_{ 41 } \\ { \sigma  }_{ 12 } & { \sigma  }_{ 2 }^{ 2 } & { \sigma  }_{ 23 } & { \sigma  }_{ 24 } \\ { \sigma  }_{ 13 } & { \sigma  }_{ 12 } & { \sigma  }_{ 3 }^{ 2 } & { \sigma  }_{ 33 } \\ { \sigma  }_{ 14 } & { \sigma  }_{ 12 } & { \sigma  }_{ 12 } & { \sigma  }_{ 4 }^{ 2 } \end{matrix} \right] 
\]

\subsubsection{Varying means, varying variances, and covariances fixed
to 0 (model
3)}\label{varying-means-varying-variances-and-covariances-fixed-to-0-model-3}

This model corresponds to the mclust model ``VVI'' and allows for the
variances to be freely estimated across profiles. The covariances are
constrained to zero. Thus, it is more flexible (and less parsimonious)
than model 1, but in terms of the covariances, is more constrained than
model 2.

\[ 
\left[ \begin{matrix} { \sigma  }_{ 1p }^{ 2 } & 0 & 0 & 0 \\ 0 & { \sigma  }_{ 2p }^{ 2 } & 0 & 0 \\ 0 & 0 & { \sigma  }_{ 3p }^{ 2 } & 0 \\ 0 & 0 & 0 & { \sigma  }_{ 4p }^{ 2 } \end{matrix} \right] 
\]

\subsubsection{Varying means, varying variances, and equal covariances
(model
4)}\label{varying-means-varying-variances-and-equal-covariances-model-4}

This model, which specifies for the variances to be freely estimated
across the profiles and for the covariances to be estimated to be equal
across profiles, extends model 3. Unfortunately, this model cannot be
specified with mclust, though it can be with MPlus; this model
\emph{can} be used with the functions to interface to MPlus described
below.

\[
\left[ \begin{matrix} { \sigma  }_{ 1p }^{ 2 } & { \sigma  }_{ 21 } & { \sigma  }_{ 31 } & { \sigma  }_{ 41 } \\ { \sigma  }_{ 12 } & { \sigma  }_{ 2p }^{ 2 } & { \sigma  }_{ 23 } & { \sigma  }_{ 24 } \\ { \sigma  }_{ 13 } & { \sigma  }_{ 12 } & { \sigma  }_{ 3p }^{ 2 } & { \sigma  }_{ 33 } \\ { \sigma  }_{ 14 } & { \sigma  }_{ 12 } & { \sigma  }_{ 12 } & { \sigma  }_{ 4p }^{ 2 } \end{matrix} \right] 
\]

\subsubsection{Varying means, equal variances, and varying covariances
(model
5)}\label{varying-means-equal-variances-and-varying-covariances-model-5}

This model specifies the variances to be equal across the profiles, but
allows the covariances to be freely estimated across the profiles. Like
model 4, this model cannot be specified with mclust, though it can be
with MPlus. Again, this model \emph{can} be used with the functions to
interface to MPlus described below.

\[
\left[ \begin{matrix} { \sigma  }_{ 1 }^{ 2 } & { \sigma  }_{ 21p } & { \sigma  }_{ 31p } & { \sigma  }_{ 41p } \\ { \sigma  }_{ 12p } & { \sigma  }_{ 2 }^{ 2 } & { \sigma  }_{ 23p } & { \sigma  }_{ 24p } \\ { \sigma  }_{ 13p } & { \sigma  }_{ 12p } & { \sigma  }_{ 3 }^{ 2 } & { \sigma  }_{ 33p } \\ { \sigma  }_{ 14p } & { \sigma  }_{ 12p } & { \sigma  }_{ 12p } & { \sigma  }_{ 4 }^{ 2 } \end{matrix} \right] \quad 
\]

\subsubsection{Varying means, varying variances, and varying covariances
(model
6)}\label{varying-means-varying-variances-and-varying-covariances-model-6}

This model corresponds to the mclust model ``VVV''. It allows the
variances and the covariances to be freely estimated across profiles.
Thus, it is the most complex model, with the potential to allow for
understanding many aspects of the variables that are used to estimate
the profiles and how they are related. However, it is less parsimonious
than all of the other models, and the added parameters should be
considered in light of how preferred this model is relative to those
with more simple specifications.

\[
\left[ \begin{matrix} { \sigma  }_{ 1p }^{ 2 } & { \sigma  }_{ 21p } & { \sigma  }_{ 31p } & { \sigma  }_{ 41p } \\ { \sigma  }_{ 12p } & { \sigma  }_{ 2p }^{ 2 } & { \sigma  }_{ 23p } & { \sigma  }_{ 24p } \\ { \sigma  }_{ 13p } & { \sigma  }_{ 12p } & { \sigma  }_{ 3p }^{ 2 } & { \sigma  }_{ 33p } \\ { \sigma  }_{ 14p } & { \sigma  }_{ 12p } & { \sigma  }_{ 12p } & { \sigma  }_{ 4p }^{ 2 } \end{matrix} \right] 
\]

\subsection{Appendix E: Additional details on the model selection
process}\label{appendix-e-additional-details-on-the-model-selection-process}

Looking across the statistics presented, some general ideas about which
models are to be preferred emerge. Solutions are interpreted first for
each model individually and then across models with the goal of choosing
a smaller number of models to investigate in more detail.

\begin{landscape}\begin{table}

\caption{\label{tab:printing-solutions-spec-stats}Solutions for models that converged with replicated LL}
\centering
\resizebox{\linewidth}{!}{\begin{tabular}[t]{rrrrrrrlll}
\toprule
Number of Profiles & LL & AIC & BIC & SABIC & CAIC & Entropy & VLMR & LMR & BLRT\\
\midrule
\addlinespace[0.3em]
\multicolumn{10}{l}{\textbf{Model 1}}\\
\hspace{1em}2 & -19894.14 & -19894.14 & 39916.16 & 39865.32 & 39820.47 & 0.807 & 3468.199 (0) & 3397.353 (0) & 3468.199 (0)\\
\hspace{1em}3 & -19453.38 & -19453.38 & 39082.59 & 39012.69 & 38951.11 & 0.794 & 881.519 (0.0126) & 863.512 (0.0136) & 881.519 (0)\\
\hspace{1em}4 & -19196.33 & -19196.33 & 38616.44 & 38527.47 & 38449.21 & 0.811 & 514.107 (0) & 503.605 (0) & 514.107 (0)\\
\hspace{1em}5 & -18817.93 & -18817.93 & 37907.60 & 37799.57 & 37704.68 & 0.913 & 756.788 (0) & 741.329 (0) & 756.788 (0)\\
\hspace{1em}6 & -18648.78 & -18648.78 & 37617.26 & 37490.17 & 37378.70 & 0.888 & 338.296 (0) & 331.386 (0) & 338.296 (0)\\
\hspace{1em}7 & -18407.23 & -18407.23 & 37182.11 & 37035.95 & 36907.95 & 0.886 & 523.141 (0.0112) & 512.455 (0.0121) & 523.141 (0)\\
\hspace{1em}9 & -18186.35 & -18186.35 & 36836.25 & 36651.96 & 36491.06 & 0.899 & 171.674 (0.1322) & 168.167 (0.1359) & 171.674 (0)\\
\addlinespace[0.3em]
\multicolumn{10}{l}{\textbf{Model 2}}\\
\hspace{1em}2 & -19107.73 & -19107.73 & 38423.27 & 38340.65 & 38267.95 & 0.924 & 850.304 (0) & 832.934 (0) & 850.304 (0)\\
\hspace{1em}3 & -18897.06 & -18897.06 & 38049.88 & 37948.20 & 37858.85 & 0.880 & 421.343 (0) & 412.736 (0) & 421.343 (0)\\
\hspace{1em}4 & -18659.68 & -18659.68 & 37623.06 & 37502.32 & 37396.37 & 0.922 & 474.773 (0) & 465.075 (0) & 474.773 (0)\\
\hspace{1em}5 & -18474.83 & -18474.83 & 37301.33 & 37161.52 & 37039.03 & 0.901 & 304.938 (0) & 298.709 (0) & 304.938 (0)\\
\bottomrule
\end{tabular}}
\end{table}
\end{landscape}

\begin{figure}

{\centering \includegraphics[width=0.5\linewidth]{rosenberg-dissertation_files/figure-latex/model1-1} 

}

\caption{Fit statistics for model 1 solutions}\label{fig:model1}
\end{figure}

\begin{figure}

{\centering \includegraphics[width=0.4\linewidth]{rosenberg-dissertation_files/figure-latex/model2-1} 

}

\caption{Fit statistics for model 2 solutions}\label{fig:model2}
\end{figure}

For solutions associated with model 1, the decrease (indicating a
preferred model) in information criteria becomes smaller as the number
of profiles increases from 5 to 6 and 6 to 7. A solution associated with
8 profiles did not replicate the log-likelihood and the VLMR and LMR
suggest that the solution associated with 9 profiles did not fit better
than that with 8 profiles, suggesting that models with 7 or fewer
profiles be preferred. Considering these models, the entropy statistic
increases by a large amount between the solution associated with 4 and 5
profiles (and then decreases slightly between 5 and 6 and 6 and 7
profile solutions), suggesting (but not providing conclusive evidence)
that models 5, 6, or 7 may be preferred. The bootstrapped LRT suggests
that, until the log-likelihood is not replicated, every more complex
model be selected. Taking these pieces of evidence into conclusion, for
model 1, solutions associated with 4 through 7 may be considered in more
depth, with an emphasis on solutions associated with profiles with 5 and
6 profiles on the basis of the slowing of the decrease in the
information criteria associated with the solutions with greater profiles
than these, and the increase in the entropy from 4 to 5 (and 6) profile
solutions.

For solutions associated with model 2, only those associated with 2-5
profile solutions were associated with log-likelihoods that were
replicated. For these four models, the log-likelihood decreased in a
mostly consistent way, such that changes in the decrease are not as
evident as those associated with model 1. The entropy statistic
decreases from 2 to 3 profile solutions, increases from 3 to 4 profile
solutions, and then decreases slightly from 4 to 5 profile solutions,
providing some information that models associated with 4 profiles be
preferred to the others. All of the LRTs suggest that the more complex
model be selected, not providing clear information about which solutions
are to be preferred. On the basis of these pieces of evidence, models
with 3, 4, and 5 solutions may be considered in more depth. However,
there is a lack of consistent evidence favoring more or less complex
models.

The model 1, six and seven profile solutions are compelling because both
show profiles that are distinguished by dimensions of engagement and its
conditions (challenge and competence). Note that for this model, only
the means and variances are estimated (and so no covariances are
estimated), and the variances are constrained to be the same across the
profiles. While this is a very restrictive model, it, along with the
model 3 type (which did not lead to solutions for any of the numbers of
profiles specified) also is a standard model for LPA, in that it meets
the assumption of local independence (of the variables that make up the
profiles--unlike for models in which covariances are estimated) typical
common to LPA (see Muthen \& Muthen, 2016). While some of the solutions
associated with the model 2 type did reach solutions, these demonstrated
less appealing properties in terms of their fit statistics as well as
their interpretability and with respect to concerns of parsimony. Thus,
while no covariances are estimated for the model 1 type solutions, there
is no requirement that these be specified; their benefit, when models
associated with them are preferred, is that they can provide better fit:
they can be used to better explain or predict the data in a sample, but
their inclusion also means that over-fitting the model to the data can
become a greater concern.

For each solution, alternate solutions associated with higher
log-likelihoods were explored. One advantage of the six profile solution
is that most of its profiles can also be identified in solutions with
fewer profiles. For the six profile solutions, this alternate solution
was very different, whereas for the seven profile solutions, this
alternate solution was highly similar. The model solutions exhibit a
less clear pattern in terms of which profiles appear when. All else
being equal, on the basis of parsimony, the model 1, six profile
solution is preferred and was selected for use in subsequent analyses.

\subsection{Appendix F: Alternate model selected (model type 1, seven
profile
solution)}\label{appendix-f-alternate-model-selected-model-type-1-seven-profile-solution}

This solution is characterized by:

\begin{itemize}
\tightlist
\item
  A \emph{full} profile, profile 7
\item
  A \emph{universally low} profile, profile 1
\item
  A \emph{competent but not engaged or challenged} profile, profile 2,
  characterized by high competence and moderate (low) or low levels of
  engagement and challenge
\item
  A \emph{moderately low} profile, profile 3, characterized by
  moderately low levels of all of the variables
\item
  A \emph{challenged} profile, profile 4, characterized by high
  challenge, moderate (high) levels of engagement, and moderate (low)
  levels of competence
\item
  A \emph{highly challenged} profile, profile 5, characterized by
  patterns similar to those of the challenged profile, but with higher
  challenge and with low levels of both engagement and challenge
\item
  A \emph{challenged but not engaged or competent} profile, profile 6,
  characterized by low levels of challenge, and high levels of
  engagement and competence
\end{itemize}

\begin{center}\includegraphics[width=0.9\linewidth]{rosenberg-dissertation_files/figure-latex/m1_7p-1} \end{center}

\begin{center}\includegraphics[width=0.9\linewidth]{rosenberg-dissertation_files/figure-latex/m1_7p-2} \end{center}

The number of observations associated with each of the profiles is not
very balanced, with few (\emph{n} = 181) observations associated with
the universally low profile and few (\emph{n} = 222) observations
associated with the highly challenged profile. The number of
observations associated with the other profiles ranged from 317 to 651.
Distinct from other solutions, none of the other five profiles were
found in the other model 1 solutions. Two pairs of the
profiles--challenged and highly challenged and universally low and
moderately low--exhibited similar patterns among the variables that were
distinguished by different mean levels. The log-likelihood was
replicated twice, with the next lowest log-likelihood being replicate
four times, possibly warranting further investigation. Taken together,
this solution raises questions about whether it may be too complex,
possibly suggesting preference for model 1 five and six profile
solutions.

\bibliography{book.bib,packages.bib}


\end{document}
